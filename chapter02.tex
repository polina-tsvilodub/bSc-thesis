Intuitively, the meaning of the positive form of a gradable adjective, for instance ‘big’, can be described as something like “big refers to the size of an object, and the size of that object that is described as big must be at least X, such that it counts as big, relative to some standard of comparison”. This means, gradable adjectives convey a feature, like size, and the degree to which the referent possesses this feature must exceed some threshold X for the referent to be felicitously described by the respective gradable adjective \parencite{Kennedy2007}. At the same time this threshold X can vary across contexts or categories: the minimal size of a flower that counts as big is quite different from the minimal size of a (real-life) truck that counts as big. Moreover, this threshold can vary within categories: the minimal size of a big sunflower is different from the minimal size of a big daisy, although both belong to the basic-level category flowers. Hence, this threshold X is heavily influenced by the set relative to which the object is compared - namely the comparison class. \\
That is, gradable adjective are vague - their meaning is subject to contextual variability, and to other characteristic features of vagueness: there exist so-called borderline cases, and these adjectives give rise to the Sorites paradox \parencite{Kennedy2007}. Specifically, even when a comparison class is set, there are cases where it is unclear whether an object counts as e.g. ‘expensive’: while a cup of coffee for \$1 is clearly cheap, and a cup for \$5 is clearly expensive, it might be difficult to say whether a  \$3.75 coffee is expensive or not - this is a borderline case. Using the same example, the Sorites paradox can be illustrated for gradable adjectives as follows: 
\begin{quotation}
P1: A \$5 cup of coffee is expensive (for a cup of coffee). 

P2: Any cup of coffee that costs 1 cent less than an expensive one is expensive (for a cup of coffee). 

C: Therefore, any free cup of coffee is expensive. 
\end{quotation}

It is the vague nature of gradable adjectives that makes it difficult to pinpoint why exactly people accept the premises so easily, and although the argument seems valid, the conclusion is clearly false (\textcite{Kennedy2007}, see for more details)
While investigating these important properties in greater detail is outside of the scope of this work, they exemplify here that representing this kind of implicit comparison to a threshold X which occurs in the positive form of gradable adjectives, while accounting for the existence of characteristics like borderline cases and eliciting the Sorites paradox is a difficult realm, which has posed difficulties to semantic theories. 

The following sections review state-of-art representations of gradable adjective semantics and the role of comparison classes therein. Then, a deeper analysis of comparison classes and prior related theoretical and experimental work is presented. 

\section{Representation of gradable adjectives}
Currently standard theories of gradable adjectives converge in representing gradable adjectives as a function mapping their argument - the referent - to a degree on an ordered scale representing some feature (e.g., ‘big’ and ‘small’ represent size),  utilizing degree morphology \parencite{Kennedy2007}. Degree morphology for positive forms of relative adjectives is informed by their comparative form, where the degree of a feature of the referent is explicitly compared to another degree of the same feature, and this comparison is overtly realised by a degree morpheme -er. For instance, in the comparative sentence ‘Bob is taller than Alice’ Bob’s height is explicitly compared to Alice’s height, expressed by the morpheme ‘-er’ appended to tall. 
By contrast, unmodified positive forms of relative adjectives which are the focus of this work don’t have an overt degree morpheme specifying the comparison; in the currently widely accepted approach (reviewed by \cite{Kennedy2007}) a phonologically silent null degree morpheme pos is introduced for this purpose. 
The morpheme pos takes the adjective as an argument and returns a standard of comparison - the context-dependent threshold (X). In \cite{Kennedy2007} the comparison class is assumed to be a property of the adjective, potentially restricting the domain of entities it applies to - an assumption discussed further in section 2.2. Formally, pos denotes the following:
%[[Deg pos λg λx. g(x)]]  = λg.λx: g(x)> s(λx: g(x) ) 
$$[[Deg_{pos} \lambda g \lambda x. g(x)  ]] = \lambda g. \lambda x: g(x) > s(\lambda x: g(x))$$

In other words, the degree to which the referent x possesses the property denoted by the adjective g must exceed some threshold, provided by s(g), where s is “a context-sensitive function that chooses a standard of comparison in such a way as to ensure that the objects that the positive form [of the adjective] is true of ‘stand out’ in the context of utterance, relative to the kind of measurement that the adjective [i.e. g] encodes” (\cite{Kennedy2007}, p.X). The contextually relevant aspects providing the threshold can be summarised as the comparison class of the adjective. 
For example, the expression ‘big dog’ is true if the size of a target dog exceeds some size-threshold, set by the comparison class. Depending on the context and the comparison class this threshold might vary: the minimal size the dog has to have in order to be described as ‘big’ is different if the dog is a toy dog, and the comparison class are other toys, than for a dog that is a Great Dane and the comparison class is other Great Danes.

Alternative to the degree-semantics framework, delineation-based formalizations of gradable adjectives treat them as unary predicates, forming partial functions depending on contextually provided comparison classes \parencite{Klein1980}. Such an approach removes degree representations from the semantics, although degrees arguably are an indispensable part of the meaning of gradable adjective \parencite{Solt2009}. 
The general issue of outlined semantics of gradable adjectives is that they assume the relevant comparison class to be supplied contextually, omitting specifying what exactly the comparison class is or how it is determined. 
While this work assumes a degree-based formalisation, it should be noted that alternative approaches also rely on the notion of contextually appropriate comparison classes, making the question addressed in this work as to how exactly comparison classes are determined a relevant one across different semantic representations.

\section{Understanding Comparison Classes}
Comparison classes can be understood as sets of entities, or reference frames the referent is compared against (\cite{Bierwisch1989}; \cite{Solt2009}; \cite{Klein1980}). In the outlined examples they were assumed to be sets of physical objects like dogs or toys. But comparison classes need not be comprised of individuals or objects, they can also comprise events or situations: In the utterance “The store is crowded for a Tuesday” the fullness of a particular store is naturally compared to other Tuesdays than to other stores \parencite{Solt2009}. Crucial is that “the comparison class provides statistical information that serves to determine the thresholds [...., and] what is relevant is not only the central value but also some measure of the extend of dispersion of values corresponding to members of the comparison class” (\cite{Solt2009} p.193).  
This naturally leads to the question is how exactly the standard of comparison - the threshold X - is determined by a given comparison class. For instance, \cite{Cresswell1976} suggested that the threshold X is the average of the relevant feature over the comparison class, but arguments have been laid against this idea, showing that these thresholds do not seem to comprise a single point on the degree scale, but should rather be represented as comprising a range of values (\cite{Kennedy2007}; \cite{vonStechow1984}). 
\cite{Solt2009} proposed that this range is computed as….

Comparison classes can be overtly expressed using prepositional for-phrases, for instance, as in “That Great Dane is big for a dog” or in “That shirt is big for you”. In the first example, additionally to expressing the comparison class, the for-phrase acts as a presupposition trigger, implying that the Great Dane is also a dog. Notably, this is not the case for the second utterance. 
\cite{Kennedy2007} suggested that for-phrases introduce a domain restriction on the gradable adjective via direct composition, hence being an argument of the adjective. That is, the comparison class restricts the domain of entities the adjective applies to. But this approach has difficulties accounting for cases when it is not the subject of the sentence that combines with the gradable adjective, or when adjectives appear in what has been labeled by \cite{ebeling1994children} as functional uses, e.g., “That short is big for you” (cf. Solt, 2011). An alternative is to interpret for-phrases in relation to the pos-morpheme, as marking its scope, similar to the relation between than-phrases and the comparative morpheme -er. In order to account for their presupposition-triggering behavior, the pos-morpheme is then assumed to take a comparison class as an argument, which by presupposition the referent is a member of (Solt, 2011). 
Formally: ….
This view follows the proposal by \cite{bartsch1973}, wherein the comparison class is an argument of a function computing the standard of comparison, whatever the nature of this function may be.   
Overt for-phrases may provide this argument in cases like “John is tall for a gymnast”, but for cases like “Sara reads difficult books for an 8-year-old” \cite{Solt2009} incidentally assumed ‘books’ to be the basis of the comparison class, rather focusing on the representation of the presupposition triggered by the for-phrase. In cases where no for-phrase is used, it is assumed that the argument of pos is a contextually appropriate implicit comparison class, e.g., supplied by the nominal modified by the adjective. Many assume this to be universally the case (cf. \cite{Cresswell1976}; \cite{Kamp1975}; \cite{Heim2000}), while \cite{Solt2009} restricts this mechanism to cases where the pos-morpheme is interpreted without raising. This leaves open the origin of the comparison class argument in sentences where the adjective appears predicatively, without a for-phrase - a question focused on in sections 2.3ff. . 
Finally, another approach to comparison class representation proposes that they “restrict binary relations, and these binary relations form the basis for the construction of [degree] scales [..., which] serve to relativize the calculation of a standard” of comparison (\cite{Bale2011}, p. 170). This proposal is based on deriving scales described by gradable adjectives from quasi-orders, i.e., those binary relations, for instance by creating so-called equivalence classes (sets of objects with equivalent degrees on that scale), which then are ordered based on the original quasi-ordering, and finally by defining a measure function via mapping each element onto its equivalence class in the scale \parencite{Bale2011}. The comparison class then restricts the quasi-order before the formation of the scale, restricting the quasi-orders to “ordered pairs consisting only of members of the comparison class”, such that the scale only describes degrees of members of the comparison class (\cite{Bale2011}, p. 178). This structure is then passed as an argument to some function returning the standard of comparison, analogous to approaches described above. One feature of this approach is the possibility to introduce a scale for gradable adjectives which are not inherently connected to some metric scale, e.g., for adjectives like beautiful or interesting \parencite{Bale2011}.   
This work focuses on the determination of the relevant comparison class, more so than on a specific semantic representation, so no commitment to a specific compositional approach shall be made here. 

Furthermore, comparison classes have been addressed from a developmental and psychological perspective. \cite{ebeling1994children} distinguish three prominent uses of gradable adjectives and comparison classes, respectively, which can be loosely related to distinct linguistic uses of those adjectives, namely occurrences of adjectives where the comparison class is supplied normatively, perceptually or functionally. 
Normative comparison class uses are based on a mental representation of the referent, for example it can comprise general world knowledge about the kind of things the referent belongs to. 
Perceptual comparison class occurrences are based on other objects of the same type as the referent physically co-present at the moment of utterance \parencite{ebeling1994children}. But the notion of perceptual comparison classes can naturally be extended to incorporate perceptually co-present objects in general (e.g. ‘That’s a big elephant’ could be said in a zoo intending to compare the elephant to other kinds of animals).
Finally, functional comparison classes are established with respect to the intended use of the referent (\cite{ebeling1994children}; \cite{sera1987}). 
While functional comparison classes are an exception in that they are very often stated overtly via a prepositional for-phrase (e.g. in ‘The shoes are too heavy for running’), both conceptual and perceptual comparison classes often remain implicit, left to the listener to infer from their world knowledge or relevant contextual aspects. This underspecification and context-dependence makes switching to these covert comparison classes form overt ones more difficult for small children, indicating the complex nature of gradable adjective interpretation \parencite{ebeling1994children}.

From a functional perspective, it can be noted that comparison classes comprising more general categories yield a wider distribution of property values than more category-specific comparison classes, yielding the latter generally more informative with respect to the meaning of the adjective. Hence, one could expect listeners assuming informative speakers to generally infer more informative - i.e., more restricted - comparison classes \parencite{tessler2017warm}.  However, this expectation might also trade-off with other pragmatic factors like typicality considerations (...) or a basic-level bias (discussed below in greater detail). 

While the notion of gradable adjectives as interpreted in reference to a comparison class has a long tradition (e.g., \cite{bartsch1973}; \cite{Bierwisch1989}), most of research in this domain has eschewed the question how exactly the relevant comparison class is identified, instead focusing on gradable adjective representation, once the respective comparison class is set (\cite{Kennedy2007}; \cite{Solt2009}; Kennedy, 2007; \cite{Kamp1975}).

\section{Compositional Accounts of Gradable Adjective Interpretation}
One line of prior work on how comparison classes might be determined approaches this question from a purely compositional perspective. In particular, the noun the adjective combines with is said to be at least a very salient contextual cue towards the comparison class (Kamp, 1975). 
Simple compositional accounts propose that the noun syntactically modified by the adjective necessarily stipulates the comparison class, such that ‘small watch’ resolves to ‘the watch is small for a watch’ (Kamp, 1975; Cresswell, 1976).
Yet, a lot of examples have been laid against such a simple mapping of the modified noun to the comparison class: intuitively, ‘John is a rich Fortune 500 CEO’ doesn’t mean that he is rich for a Fortune 500 CEO; ‘Kyle’s car is an expensive BMW’ doesn’t mean that his car is expensive relative to other BMWs (Kennedy, 2007). Furthermore, such a theory does not account for the interpretation of adjectives occurring predicatively, whereas a lot of adjectives - gradable adjectives in particular - can occur in either position (for example, attributively: ‘That’s a big dog’; or predicatively : ‘That dog is big’; cf. \cite{mcnally2008}; \cite{hofherr2010adjectives}).
The simplest generalisation of these compositional accounts to predicative adjectives would posit that the noun of the sentence generally sets the comparison class, such that the utterance “That Great Dane is big” would be taken to mean “That Great Dane is big for a Great Dane” (Tessler et al., to appear).
The exact syntactic relation between the two kinds of adjectives is widely discussed (e.g., \cite{Cresswell1976}; \cite{Kamp1975, hofherr2010adjectives}); however, the different potential formalisations would not affect the core of the presented proposal, since gradable adjective interpretation is approached from the perspective of the noun position rather than the adjective position, wherein the syntax only provides cues to high-level informational goals, as opposed to stipulating specific low-level syntactic relations.

Another distinction along the lines of adjective position potentially relevant for comparison class determination is the difference between intersective and non-intersective (or subsective) adjective readings \parencite{kennedy2012, hofherr2010adjectives}. 
Intersective adjective interpretations emerge when the target is interpreted as a member of the intersection of two sets: the one denoted by the noun and the one denoted by the adjective \parencite{kennedy2012}. For example, the sentence ‘Bobby is an expensive dog’ implies that Bobby is both expensive (in general) and that he is a dog. 
In contrast, subsective interpretations emerge when the referent is interpreted as a member of a subset of the set denoted by the noun, returned by the adjective combining with the noun: For example, the sentence “John is a skillful surgeon” implies that he is a surgeon, but not necessarily that he is generally skillful - it only implies that he is skillful as a surgeon \parencite{kennedy2012}. Nonsubsective adjectives like ‘former’ can be disregarded for our purposes. 
Although it is argued in the literature that specifically prenominal attributive adjectives give rise to ambiguity between the two readings (cf. ‘Olga is a beautiful dancer’; \cite{hofherr2010adjectives}), it seems plausible to a priori treat gradable adjectives occuring in either position (attributive: ‘That’s a big dog’, or predicative: ‘That dog is big’) as eliciting intersective interpretations (i.e. both amount to the reading ‘the referent is both a dog and is generally big’), therefore leaving the comparison class underspecified. Positing a subsective reading for either sentence frame (i.e. ‘the referent is big for a dog, but not generally big’) would amount to the simple syntactic hypothesis wherein the noun sets the compison class, which intuitively can not hold in general (especially given examples like “John is a rich Fortune 500 CEO”: positing a subsective reading would translate to the sentence “John is rich for a Fortune 500 CEO, but not rich in general”, which intuitively isn’t correct). The ambiguity mentioned above would then rather be ascribed to the comparison class ambiguity, than to the ambiguity of the denotation set. 

Syntactic accounts outlined above stipulate that the meaning of an utterance involving gradable adjectives is fully specified by its words. Yet it was shown that several other pragmatic components like context of the utterance and listeners’ world knowledge seem to come together when gradable adjectives are interpreted. Psycholinguistic studies investigated the role of these pragmatic factors for gradable adjectives and comparison class determination. 

\section{Pragmatic Accounts of Gradable Adjective Interpretation}
Several visual world studies addressed the role of context for relative adjective interpretation, specifically focusing on manipulating the distractor set and the feature degrees of the distractors in the context in which a target described by an adjective was presented (e.g., \cite{sedivy1999, Aparicio2015}). 
They considered prenominally modified nouns in predicative position for the goal of reference resolution, showing that targets were identified faster when the context supported restrictive interpretation of the predicate (Sedivy et al., 1999; Aparicio, Xiang, Kennedy, 2015). That is, then context included another object of the same kind as the target, but with a different degree of the feature described by the adjective, such that the adjective was interpreted as helping ‘restrict’ the pair of objects to the target . By contrast, this effect was not shown in contexts including distractors of other categories than the target \parencite{Aparicio2015}. 
Notably, both studies assume that comparison classes of gradable adjectives were supplied contextually, and more so when the context was more homogenous (i.e. included several objects of the same type) (cf. \cite{Aparicio2015}), showing the context-dependent nature of comparison class determination. 

Another recent study addressed the role of world knowledge for comparison class inference \parencite{tessler2017warm}. The authors showed that listeners flexibly adjust comparison classes of gradable adjectives based on their world knowledge, when encountering simple utterances like “He’s tall” said of targets about which listeners typically have strong expectations regarding the degree of the feature described by the adjective. They showed that listeners are more likely to infer that an utterance like “He’s tall” said of a basketball player means “He’s tall for a person” (i.e. relative to a general, basic-level category), whereas the utterance “He’s short” rather means “He’s short for a basketball player” (i.e. relative to the target’s subordinate category). This pattern was clearly shown for targets of those categories which exhibit a rather high or low degree of the feature (e.g. basketball players, whose height is generally quite large; and jockeys, whose height is generally rather low) \parencite{tessler2017warm}. 
The present studies build on this experimental paradigm, making use of listeners’ expectation about such categories highly salient with respect to some feature. When adjectives describing this certain feature scale are attributed of those categories, the basic-level comparison class is a priori more likely to provide a felicitous expression than the subordinate comparison class. 
However, the study by \cite{tessler2017warm} only considered simple utterances, appearing without much context or a noun. 
