Absolute vs. relative gradable adjectives. 
One particularly interesting case study of context-sensitive language are gradable adjectives. Specifically, it depends on the context what exactly counts as \textit{tall, expensive, small} or \textit{full} - a one-meter tall three-year-old counts as tall, but a one-meter tall redwood tree does not; a three-quarter full cup of coffee counts as full, but a three-quarter full spaceship fuel tank does not. While both examples show context-sensitivity of the adjective's meaning, these two adjectives differ in what exactly about their meaning depends on the context: in case of \textit{relative gradable adjectives} like 'tall' the context determines how much of the feature described by the adjective is required to count as 'tall', whereas in case of \textit{absolute gradable adjectives} like 'full' the context determines how much the degree of the described feature may deviate from total fullness \parencite{Aparicio2016, Kennedy2007, hofherr2010adjectives}.  

In particular, the meaning of a relative gradable adjective, for instance 'big', can be described in that 'big' refers to the size of an object, and the size of that object  described as 'big' must be at least $X$, such that it counts as big, relative to some standard of comparison $\theta$. This means, relative gradable adjectives convey a feature, like size, and the degree to which the referent possesses this feature must exceed some threshold $\theta$ for the referent to be felicitously described by the respective gradable adjective \parencite[e.g.,][]{Kennedy2007}. At the same time this threshold $\theta$ can vary across contexts or categories: the minimal size of a flower that counts as big is quite different from the minimal size of a house that counts as big. Moreover, this threshold can vary within categories: the minimal size of a big sunflower is different from the minimal size of a big daisy, although both belong to the category flowers. Hence, this threshold $\theta$ is strongly influenced by the set relative to which the object is compared - namely the \textit{comparison class}.

In contrast, the meaning of an absolute gradable adjective, for instance 'full', is debated: some researchers argue that it refers to an endpoint on the feature scale described by the adjective, i.e., 'full' refers to the maximum on the scale of volume for the object under discussion \parencite{Kennedy2007, Aparicio2016, Qing2014}. Others argue that the meaning of absolute gradable adjectives is also resolved relative to a context-sensitive threshold $\theta$, by mechanisms universal for all gradable adjectives \parencite{lassiter2017adjectival}.

Generally, gradable adjectives are \textit{vague} - their meaning is subject to contextual variability, and to other characteristic features of vagueness: there exist so-called borderline cases, and these adjectives give rise to the Sorites paradox \parencite{Kennedy2007}. Specifically, even when a comparison class is set, there are cases where it is unclear whether an object counts as e.g. ‘expensive’: while a cup of coffee for \$1 is clearly cheap, and a cup for \$5 is clearly expensive, it might be difficult to say whether a  \$3.75 coffee is expensive or not - this is a borderline case. Using the same example, the Sorites paradox can be illustrated for gradable adjectives as follows: 
\begin{quotation}
P1: A \$5 cup of coffee is expensive (for a cup of coffee). 

P2: Any cup of coffee that costs 1 cent less than an expensive one is expensive (for a cup of coffee). 

C: Therefore, any free cup of coffee is expensive. 
\end{quotation}

It is the vague nature of gradable adjectives that makes it difficult to pinpoint why exactly people accept the premises so easily, and although the argument seems valid, the conclusion is clearly false \parencite[see][for more details]{Kennedy2007}.

Investigating these important properties in greater detail is outside of the scope of this work: in the remainder, the focus is to investigate the importance of comparison classes, specifically for relative gradable adjectives. Yet characteristics like borderline cases and eliciting the Sorites paradox emphasize that capturing the kind of implicit comparison to a threshold $\theta$ which occurs in the positive form of gradable adjectives, while accounting for the existence of these properties is rather difficult. 
The following sections review state-of-art representations of relative gradable adjective semantics and the role of comparison classes therein. Then, prior related theoretical and experimental work on comprison classes is presented. 

\section{Semantic Representation of Gradable Adjectives}
\label{2.1.}
Currently standard theories of gradable adjectives converge in representing gradable adjectives as a function mapping their argument - the referent - to a degree on an ordered scale representing some feature (e.g., ‘big’ and ‘small’ represent size),  utilizing degree morphology \parencite{Kennedy2007}. Degree morphology for positive forms of relative adjectives is informed by their comparative form, where the degree of a feature of the referent is explicitly compared to another degree of the same feature, and this comparison is overtly realised by a degree morpheme \textit{-er}. For instance, in the comparative sentence ‘Bob is taller than Alice’ Bob’s height is explicitly compared to Alice’s height, expressed by the morpheme \textit{-er} appended to tall. 
By contrast, unmodified positive forms of relative adjectives which are the focus of this work don’t have an overt degree morpheme specifying the comparison to some point of reference; in the currently widely accepted approach (reviewed by \cite{Kennedy2007}) a phonologically silent null degree morpheme \textit{pos} is introduced for this purpose. 
The morpheme \textit{pos} takes the adjective as an argument and returns a standard of comparison - the context-dependent threshold $\theta$. In \textcite{Kennedy2007}, the comparison class is assumed to be an argument of the adjective, potentially restricting the domain of entities it applies to - an assumption discussed further in section \ref{2.2}. Formally, \textit{pos} denotes the following:

$$\llbracket_{Deg} pos \lambda g \lambda x. g(x)  \rrbracket = \lambda g. \lambda x: g(x) > s(\lambda x: g(x))$$

In other words, the degree to which the referent $x$ possesses the property denoted by the adjective $g$ must exceed some threshold, provided by $s(g)$, where $s$ is “a context-sensitive function that chooses a standard of comparison in such a way as to ensure that the objects that the positive form [of the adjective] is true of ‘stand out’ in the context of utterance, relative to the kind of measurement that the adjective [i.e., $g$] encodes” \parencite[p. 17]{Kennedy2007}. The contextually relevant aspects providing the threshold can be summarised as the comparison class of the adjective. 
For example, the expression ‘big dog’ is true if the size of a target dog exceeds some size-threshold, set by the comparison class. Depending on the context and the comparison class this threshold might vary: the minimal size the dog has to have in order to be described as ‘big’ is different if the dog is a toy dog, and the comparison class are other toys, than for a dog that is a Great Dane and the comparison class is other Great Danes.

Alternative to the degree-semantics framework, delineation-based formalizations of gradable adjectives treat them as unary predicates, forming partial functions depending on contextually provided comparison classes \parencite{Klein1980}. Such an approach removes degree representations from the semantics, although degrees arguably are an indispensable part of the meaning of gradable adjective \parencite{Solt2009}. 

The general issue of outlined semantic representations of gradable adjectives is that they assume the relevant comparison class to be supplied contextually, yet omitting to specify what exactly the comparison class is or how it is determined. 
While this work assumes a degree-based formalisation, it should be noted that alternative approaches also rely on the notion of contextually appropriate comparison classes, making the question addressed in this work as to how exactly comparison classes are determined a relevant one across different semantic representations.

\section{Understanding Comparison Classes}
\label{2.2.}
Comparison classes can be understood as sets of entities, or reference frames the object described by the adjective is compared against \parencite{Bierwisch1989, Solt2009, Klein1980}. In the outlined examples comparison classes were assumed to be sets of physical objects like dogs or flowers. But comparison classes need not be comprised of individuals or objects, they can also comprise events or locations: In the utterance “The store is crowded for a Tuesday” the fullness of a particular store is naturally compared to other Tuesdays, rather than to other stores \parencite{Solt2009}. It is crucial that “the comparison class provides statistical information that serves to determine the thresholds [...., and] what is relevant is not only the central value but also some measure of the extend of dispersion of values corresponding to members of the comparison class” \parencite[p.193]{Solt2009}.  Note that the width of the value distribution is closely related to the specificity of the comparison class: more general categories serving a comparison classes like\textit{basic-level} categories tend to imply a wider distribution than more specific comparison classes, for instance based on \textit{subordinate} categories \parencite{rosch1976}. From a pragmatic perspective, cooperative speakers should tend to use relatively specific comparison classes appropriate in context, since they are more informative with respect to the underspecified threshold $\theta$ than more general ones. Listeners assuming cooperative speakers would then tend to infer maximally specific comparison classes, respectively \parencite{tessler2017warm}. 

This naturally leads to the question is how exactly the standard of comparison - the threshold $\theta$ - is determined by a given comparison class. For instance, \textcite{Cresswell1976} suggested that the threshold $\theta$ is the average of the relevant feature over the comparison class, but arguments have been laid against this idea, showing that these thresholds do not seem to comprise a single point on the degree scale, but should rather be represented as comprising a range of values \parencite{Kennedy2007, vonStechow1984}. 
One proposal by \textcite[p.194]{Solt2009} is that this range is computed as an interval around the median $median_{x\in C}$ provided by the comparison class $C$ (which the target referent $x$ is a member of), where the width of this interval is determined by the degree of variability of the feature in the comparison class, as provided by the measure function $MEAS$ and captured by the median absolute deviation ($MAD$):

$$R_{Std:C} = median_{x \in C} MEAS(x) \pm n \bullet MAD_{x \in C} MEAS(x)$$

However, it is still unclear how the relevant comparison class $C$ is determined. Comparison classes can be expressed overtly using prepositional \textit{for}-phrases, for instance, as in “That Great Dane is \emph{big for a dog}” or in “That shirt is \emph{big for you}”. In the first example, additionally to expressing the comparison class, the \textit{for}-phrase acts as a \textit{presupposition trigger}, implying that the Great Dane is also a dog \parencite[cf. ]{Bale2011, Solt2009}. Notably, this is not the case for the second example. 

There are several proposals with respect to compositional semantic integration of \textit{for}-phrases. \textcite{Kennedy2007} suggested that \textit{for}-phrases introduce a domain restriction on the gradable adjective via direct composition, hence being an argument of the adjective. That is, the comparison class restricts the domain of entities the adjective applies to. But this approach has difficulties accounting for cases when it is not the subject of the sentence that combines with the gradable adjective, or when adjectives appear in what has been labeled by \textcite{ebeling1994children} as \textit{functional uses}, e.g., “That short is big for you” \parencite{Solt2009}. 

An alternative is to interpret \textit{for}-phrases in relation to the \textit{pos}-morpheme, as marking its scope, similar to the relation between \textit{than}-phrases and the comparative morpheme \textit{-er}. In order to account for their presupposition-triggering behavior, the \textit{pos}-morpheme is then assumed to take a comparison class $C$ as an argument, which by presupposition the referent is a member of \parencite{Solt2009}. 
Formally:
$$\llbracket POS \rrbracket = \lambda C_{\langle et \rangle } \lambda P_{\langle d, et \rangle } \lambda x: x \in C.\forall d \in R_{Std:C}[P(x,d)]$$ 
where $P(x,d)$ denotes the measure function mapping individuals onto respective degrees on the feature scale described by the adjective, and $R_{Std:C}$ is the standard of comparison, e.g., computed as described above.
This view follows the proposal by \textcite{bartsch1972}, wherein the comparison class is an argument of a function computing the standard of comparison, whatever the nature of this function may be.  However, in cases like “John is tall for a gymnast”, overt \textit{for}-phrases may provide this argument, but for cases like “Sara reads difficult books for an 8-year-old” \textcite{Solt2009} assumed ‘books’ to be the basis of the comparison class rather incidentally, focusing on the representation of the presupposition triggered by the \textit{for}-phrase. 

Finally, another approach to comparison class representation proposes that they “restrict binary relations, and these binary relations form the basis for the construction of [degree] scales [..., which] serve to relativize the calculation of a standard” of comparison \parencite[p.170]{Bale2011}. This proposal is based on deriving scales described by gradable adjectives from quasi-orders, i.e., those binary relations, for instance by creating so-called equivalence classes (sets of objects with equivalent degrees on that scale), which then are ordered based on the original quasi-ordering, and finally by defining a measure function via mapping each element onto its equivalence class in the scale \parencite{Bale2011}. Comparison classes then restricts the quasi-order before formation of the scale, restricting the quasi-orders to “ordered pairs consisting only of members of the comparison class”, such that the scale only describes degrees of members of the comparison class \parencite[p.178]{Bale2011}. This structure is then passed as an argument to some function returning the standard of comparison, analogous to approaches described above. One feature of this approach is the possibility to introduce a scale for gradable adjectives which are not inherently connected to some metric scale, e.g., for adjectives like 'beautiful' or 'interesting' \parencite{Bale2011}.   

In cases where no overt \textit{for}-phrase is used, it is assumed that the argument of \textit{pos} is a contextually appropriate implicit comparison class, e.g., supplied syntactically by the nominal modified by the adjective. Many assume the modified noun to supply the comparison class universally \parencite[cf. e.g. ][]{Cresswell1976, Kamp1975, Heim2000}, while \textcite{Solt2009} restricts this mechanism in terms of the \textit{pos}-morpheme scope, proposing that comparison class saturation is local given a modified nominal, but involves raising in case of \textit{for}-phrases. This leaves open the origin of comparison class arguments in sentences where the adjective appears predicatively without a \textit{for}-phrase - a question focused on in sections \ref{2.3.}, \ref{2.4.}. 

This work focuses on the determination of relevant comparison classes even before they are integrated compositionally, so no commitment to a specific compositional approach shall be made here. 

Gradable adjectives and comparison classes, respectively, have also been addressed from a developmental and psychological perspective, in particular as a case study of children's developing understanding of context. 
\textcite{barner2008} have shown that by the age of 4 years, children are able to track statistical regularities of a property described by an adjective (e.g., height described by 'tall') in a novel population of toys ('pimwits') and flexibly adjust their use of the adjective according to changes of the property distribution. 
 
\textcite{ebeling1994children} distinguish three prominent uses of gradable adjectives children are exposed to, which can be loosely related to distinct linguistic constructions they tend to occur in and how the comparison class may be supplied, namely occurrences of adjectives where the comparison class is supplied \textit{normatively, perceptually} or \textit{functionally}. 
Normative comparison classes are based on a mental representation of the referent, for example it can comprise general world knowledge about the kind of things the referent belongs to. One could hypothesize that here the relevant knowledge might remain implicit and require interlocutors to infer relevant cues from context. 
Perceptual comparison classes are based on other objects of the same type as the referent physically co-present at the moment of utterance \parencite{ebeling1994children}. The notion of perceptual comparison classes could naturally be extended to incorporate perceptually co-present objects of other kinds, in general. These comparison class uses might require less implicit general knowledge, but might still require figuring out which aspects of context are relevant. 
Finally, functional comparison class uses reference the intended use of the object, as in the aforementioned example "This shirt is \emph{big for you}" \parencite{ebeling1994children, sera1987}. 
While 'functional' comparison classes may be an exception in that they are very often stated overtly via the prepositional \textit{for}-phrase, both normative and perceptual comparison classes often remain implicit, left to the listener to infer from their world knowledge or relevant contextual aspects. 
A preliminary study shows that adults might use syntactic structure of the utterance containing the adjective to signal the intended comparison class in such underspecified cases, consistent with the reference-predication trade-off hypothesis proposed in this work \parencite[discussed in greater detail in Chapter \ref{chapter06}]{sinelnikova2020}. 

\section{Semantic and Syntactic Aspects of Gradable Adjective Interpretation}
\label{2.3.}
%former 'compositional accounts' section; maybe do a prior-work chapter?

While the notion of relative gradable adjectives as interpreted in reference to a comparison class has a long tradition \parencite[e.g.,][]{bartsch1972, Bierwisch1989}, there is little agreement on how exactly relevant comparison classes are identified when not supplied overtly. Prior work reviewed in this section has mainly focused on how syntactic and semantic properties of adjectives determine them.

One line of work on how comparison classes might be determined approaches this question from a purely compositional perspective. In particular, the noun the adjective combines with is said to be at least a very salient contextual cue towards the comparison class \parencite{Kamp1975}. 
Simple compositional accounts propose that the nominal syntactically modified by the adjective necessarily stipulates the comparison class, such that ‘small watch’ resolves to ‘the watch is small for a watch’ \parencite{Kamp1975, Cresswell1976}. More sophisticated ideas involve syntactic aspects of interpreting the \textit{pos}-morpheme (see section \ref{2.2.}).
Yet, a lot of examples have been laid against such a simple mapping of the modified noun to the comparison class: intuitively, ‘John is a rich Fortune 500 CEO’ doesn’t mean that he is \emph{rich for a Fortune 500 CEO}; ‘Kyle’s car is an expensive BMW’ doesn’t mean that his car is \emph{expensive relative to other BMWs} \parencite{Kennedy2007}. 

However, such syntactic theories focus on gradable adjectives occuring attributively, not accounting for their flexibility in occuring both attributively and predicatively (for example, attributive: ‘That’s a big dog’; or predicative: ‘That dog is big’; \textcite[cf.][]{mcnally2008, hofherr2010adjectives}). Furthermore, attributive adjectives can occur prenominally (e.g., 'visible stars') and post-nominally (e.g., 'stars visible [tonight]') \parencite{hofherr2010adjectives}. In English, the common basic position of attributive adjectives is prenominal, but post-nominal in e.g. Italian \parencite{cinque2010}; for this work focusing on English, post-nominal cases will be disregarded. 
  
The exact syntactic relation between attributive and predicative occurences of adjectives is widely discussed; prior work attempted to derive one kind of syntactic construction from the other\parencite[e.g.,][]{Cresswell1976}. For instance, predicative adjectives might be seen as elliptical uses derived from underlying attributive adjectives (e.g., 'The dog is big' derived from 'The dog is a big dog', \textcite[cf.][]{Kamp1975}) or anaphoric constructions (e.g., 'The dog is big' derived from 'The dog is a big one', \textcite[cf.][]{goldberg2017one}; however, the most reasonable resolution of the anaphora would stipulate referring to the subject noun 'dog', reducing this idea to the former one). 
This implies the simplest generalisation of these compositional syntactic accounts to predicative adjectives: one would posit that the noun of the sentence generally sets the comparison class, such that the utterance “That Great Dane is big” would be taken to mean “That Great Dane is big for a Great Dane” \parencite{tessler2020}. Yet, similar intuitive counter-examples might be put forward here. 
Therefore, although the noun the adjective combines with is arguably a salient cue to the comparison class, the degree to which it restricts the comparison class might vary across different utterances and contexts.   

Alternatively, one could imagine syntactic accounts of gradable adjective interpretation wherein the presence of syntactic modification would be the critical signal towards the role of the noun for comparison class restriction. Specifically, in presence of syntactic modification (i.e., in prenominal adjectives) the modified noun would set the comparison class akin to the simple syntactic account outlined above,  while absence of modification (i.e., in predicative adjectives) would signal that the noun is \emph{not} the comparison class. However, this alternative would not resolve remarks made against the compositional account, and it would remain unclear how comparison classes are determined in absence of modification by any compositional mechanisms different from what has been outlined above. The only viable alternative then seems to involve some kind of pragmatic reasoning (e.g., considering general \textcite[world knowledge][]{tessler2017warm}), at least for the predicative cases. Chapter \ref{chapter04} suggests experimental evidence, ruling out purely compositional accounts of comparison class determination. 
%modification accounts , cf google doc table 

A semantic distinction vaguely parallel to the adjectives' syntactic positions potentially relevant for comparison class determination is the difference between \textit{intersective} and \textit{non-intersective} (or subsective) adjective readings \parencite{kennedy2012, hofherr2010adjectives}. 
Intersective adjective interpretations emerge when the target is interpreted as a member of the intersection of two sets: the one denoted by the noun and the one denoted by the adjective \parencite{kennedy2012}. For example, the sentence ‘Bobby is an expensive dog’ implies that Bobby is both expensive (in general, whatever this might mean) and that he is a dog. 
In contrast, subsective interpretations emerge when the referent is interpreted as a member of a subset of the set denoted by the noun, returned by the adjective combining with the noun: For example, the sentence “John is a skillful surgeon” implies that he is a surgeon, but not necessarily that he is generally skillful - it only implies that he is skillful as a surgeon \parencite{kennedy2012}. Nonsubsective adjectives like ‘former’ can be disregarded for purposes of this work. 
Although it is argued in the literature that specifically prenominal attributive adjectives give rise to ambiguity between the two readings \parencite[cf. ‘Olga is a beautiful dancer’][]{hofherr2010adjectives}, it seems plausible a priori to treat gradable adjectives occuring in either position (attributively or predicatively) as eliciting intersective interpretations, therefore leaving the comparison class underspecified. Positing a subsective reading for either sentence frame would amount to the simple syntactic hypothesis wherein the noun sets the compison class, which intuitively does not hold in general (especially given examples like “John is a rich Fortune 500 CEO”: positing a subsective reading would translate to the sentence “John is rich for a Fortune 500 CEO, but not rich in general”, which intuitively isn’t correct). 

To sum up, compositional syntactic accounts outlined above stipulate that the meaning of an utterance involving gradable adjectives is fully specified by its words. Yet it was shown that several other pragmatic components like context of the utterance and listeners’ world knowledge seem to have a large influence on the meaning of vague gradable adjectives \parencite[e.g.,][]{tessler2017warm, Kennedy2007}. Psycholinguistic studies investigating the role of these pragmatic factors for gradable adjectives and comparison class determination are reviewed in the next section. 

\section{Pragmatic Aspects of Gradable Adjective Interpretation}
\label{2.4.}
Several visual world studies addressed the role of context for relative adjective interpretation, specifically focusing on manipulating the distractor set and the feature degrees of the distractors in the context in which a target described by an adjective was presented \parencite[e.g.,]{sedivy1999, Aparicio2016}. 
They considered prenominally modified nouns in predicative position for the goal of reference resolution, showing that targets were identified faster when the context supported restrictive interpretation of the predicate (Sedivy et al., 1999; Aparicio, Xiang, Kennedy, 2015). That is, then context included another object of the same kind as the target, but with a different degree of the feature described by the adjective, such that the adjective was interpreted as helping ‘restrict’ the pair of objects to the target . By contrast, this effect was not shown in contexts including distractors of other categories than the target \parencite{Aparicio2016}. 
Notably, both studies assume that comparison classes of gradable adjectives were supplied contextually, and more so when the context was more homogenous (i.e. included several objects of the same type) \parencite[cf.][]{Aparicio2016}, showing the context-dependent nature of comparison class determination. 

Another recent study addressed the role of world knowledge for comparison class inference \parencite{tessler2017warm}. The authors showed that listeners flexibly adjust comparison classes of gradable adjectives based on their world knowledge, when encountering simple utterances like “He’s tall” said of targets about which listeners typically have strong expectations regarding the degree of the feature described by the adjective. They showed that listeners are more likely to infer that an utterance like “He’s tall” said of a basketball player means “He’s tall for a person” (i.e. relative to a general, basic-level category), whereas the utterance “He’s short” rather means “He’s short for a basketball player” (i.e. relative to the target’s subordinate category). This pattern was clearly shown for targets of those categories which exhibit a rather high or low degree of the feature (e.g. basketball players, whose height is generally quite large; and jockeys, whose height is generally rather low) \parencite{tessler2017warm}. 
The present studies build on this experimental paradigm, making use of listeners’ expectation about such categories highly salient with respect to some feature. When adjectives describing this certain feature scale are attributed of those categories, the basic-level comparison class is a priori more likely to provide a felicitous expression than the subordinate comparison class. 
However, the study by \textcite{tessler2017warm} only considered simple utterances, appearing without much context or a noun. 
