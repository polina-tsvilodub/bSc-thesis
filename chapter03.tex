This section aims to integrate both the role of the noun in the sentence as well as the role of pragmatic cues like perceptual context and world knowledge for relative adjective interpretation, presenting a reference-predication trade-off hypothesis of comparison class inference. 

Specifically, the issue of comparison class determination is approached from a functional perspective, based on the question what informational goals speakers might pursue when producing an utterance, and how these goals might influence listeners’ comparison class inferences \parencite{tessler2020}.
The proposed approach is an inferential account of comparison class determination, informed by the idea of recursive social reasoning mechanisms, applied to language in Gricean tradition: Speakers have basic informational goals which guide how they craft their utterance in order to facilitate the message interpretation for a listener \parencite{goodman2016}. Listeners, in turn, infer the most likely comparison class in light of those speaker goals. In particular, in order to communicate a property about a referent, speakers must achieve two goals: reference - identifying the right target - and predication - attributing a property to the target, specifically, communicating  the degree of the feature denoted by the gradable adjective (Reboul, 2001; Tessler et al, ta).  
For the two stated informational goals, it is reasonable to posit that listeners generally expect the subject to be sufficient in order to establish reference - independent of the predicate asserted to hold of the subject - and that speakers aim to satisfy this general expectation (Reboul, 2001; Tessler et al, ta). This expectation might be based on general information structural reasons: In order to predicate a property of a target, i.e. potentially provide a comment in the sense of adding some information about the target, this target must be clear (Searle, 1969; Krifka, 2008). There are exceptions to this tendency: e.g. the pronominal he can be resolved not only after applying the predicate, but also only taking into account the context of the sentence “The boss fired the worker because he was a convinced communist”: He can either refer to the boss or to the worker (Reboul, 2001). But these syntactic expectations are a general enough heuristic holding in many contexts.
These expectations have implications for comparison classes of gradable adjectives insofar as speakers have the liberty to choose from a priori similar sentence options to communicate the same message. For example, in order to tell some 4-year-old kids on a playground in winter that they built a big snowman, a speaker has the liberty to say “That’s big!” pointing at the snowman, “That snowman you built is big!”, “You built a big snowman!” and many other options. Consequently, the choice of a particular sentence over other equivalent options might respond to particular informational - communicative needs. 

From this perspective, the influence of the noun on the comparison class in a simple “Subject Predicate” sentence depends on its position in the sentence (Tessler et al, ta). If the noun appears in the predicate of the sentence (e.g. in “That’s a big Great Dane”), it can naturally be explained as produced by a speaker intending to constrain the comparison class. By contrast, if the noun appears in the subject of the sentence (e.g. in “That Great Dane is big”), it can potentially be explained away as produced by a speaker intended to support reference (especially via combining it with the deictic ‘that’), and hence it is a weaker cue towards the comparison class, leaving it to other pragmatic cues like world knowledge (Tessler et al., 2017) or perceptual context to guide the comparison class inference. 
Hence, the utility of the noun as constraining the comparison class is the result of a trade-off between reference and predication, such that the noun is integrated with syntactic and contextual cues in order to infer the comparison class (Tessler et al, ta).

\section{Understanding Reference and Predication}
This reference-predication trade-off hypothesis takes a rather linguistic point of view on phenomena treated in the literature from various perspectives. 
Searle (1969) conceptualizes both reference and predication as particular kinds of propositional acts, defining conditions to be fulfilled in order to accomplish them. Of particular importance for accomplishing reference is that the expression intended for reference isolates the target referent for the listener (Searle, 1969). Studies have shown that speakers are sensitive to contextual variability and adjust the informativity of their referential expression correspondingly, such that this requirement is satisfied (Graf et al., 2016). One of the requirements for accomplishing predication is that the same sentence contains a reference to the intended target of predication (Searle, 1969; Reboul, 2001); together, these two conditions support the presented hypothesis. 
Whereas it is debated which linguistic expressions actually establish reference (Reboul, 2001; Michaelson and Reimer, 2019), it should be noted that the trade-off hypothesis specifically takes advantage of the flexibility of nouns with respect to both informational goals: combining with the deictic ‘that’ which acceptedly is taken to be referential (Reboul, 2001; Braun, 2017), the noun can accomplish reference; but the bare deictic ‘that’ can also be sufficient for reference if supported by the context, such that the noun of the sentence may then be ‘left over’ for accomplishing predication (Reboul, 2001). 


\section{Experimental Operationalization}
In present studies, this property of nouns leads to the operationalization of the reference-predication trade-off hypothesis via a syntactic manipulation, wherein the noun (N) appears either in the subject or in the predicate of a sentence involving a gradable adjective ADJ (i.e. “That N is ADJ” (subject N), “That’s a ADJ N” (predicate N); Experiment 1 - Experiment 3) (Tessler et al, ta). 
The critical question addressed by this distinction is how speakers and listeners treat these syntactic frames, asserting the ADJ of referents for whom they are felicitous given one comparison class, but not another (e.g. a normal-sized Great Dane can felicitously be described as ‘big’ given the comparison class ‘dogs’, but not ‘Great Danes’) (Tessler et al, ta). The reference-predication trade-off hypothesis predicts that nouns that are more likely to establish reference are less likely to constrain the comparison class. Conversely, when the noun is taken to contribute to predication, the noun is rather expected to be consistent with the comparison class felicitous in order to describe the target: i.e. the basic-level category label is more appropriate for setting the comparison class when describing a normal-sized target from a large-subordinate category as ‘big’, than the subordinate category label (cf. Tessler et al., 2017). 

Note that although the differences in comparison class inference is approached through the lense of this syntactic manipulation, the underlying communicative goals are the primary driving force in comparison class inference, to which the syntax is just a cue, as supported by exploratory analyses of Experiment 3 (see Section 4; Tessler et al, ta) and Experiment 4. 
There might well be other syntactic realisations of these informational goals (Reboul, 2001): The sentence “What is big is that Great Dane” seems appropriate in a context where generally big things are discussed; in this utterance reference is accomplished from the predicate, and because of this referential pressure, under the trade-off hypothesis the noun would not be expected to constrain the comparison class, although it appears in the predicate, supporting the view that the syntactic position of the noun is dissociable from the intended communicative goals.

To show that informational goals are primary for comparison class inference as opposed to specific syntactic properties of the adjectival phrase, another syntactic manipulation is employed in the fourth experiment wherein the noun appearing in the subject or predicate of the sentence is always syntactically modified by the adjective. This manipulation allows to disentangle the effect of the noun position from the effect of syntactic modification of the noun. For example, the critical sentences are “That big Great Dane is a prize-winner” (subject-N) or “That prize-winner is a big Great Dane” (predicate-N). The trade-off hypothesis predicts that even directly modified nouns in the subject position contribute to reference, and thus should be less likely to constrain the comparison class.

This work presents four behavioural experiments exploring the reference-predication trade-off hypothesis, specifically investigating the use of the size adjectives ‘big’ and ‘small’. These two adjectives are chosen for practical reasons: size is a visually accessible feature, allowing for easy presentation and manipulation of the context in web-based experiments. Furthermore, humans usually have strong expectations about typical size distributions of different natural categories, from which the target referents were sampled for the experiments. Three distinct dependent measures were used to assess the influence of various cues on comparison class inference. 
Finally, the hypothesis is formalized computationally within the Rational Speech Act framework (Frank and Goodman, 2012). Extending a state-of-art model accounting for flexible use of comparison classes (Tessler et al., 2017), the model incorporates reasoning about the most likely comparison class via reasoning about the referential utility of possible utterances in context, aiming to qualitatively fit experimental data. 

