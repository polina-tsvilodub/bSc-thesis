The meaning of natural language expressions heavily depends on the context in which these expressions are used, but speakers rarely explicitly outline which aspects of the context are relevant for their interpretation. 
	This issue is clearly illustrated by utterances involving gradable adjectives like big, small, tall, expensive etc. These adjectives are typically taken to describe a degree to which an object possesses some property, e.g., the degree of tallness (i.e., height) for the adjective ‘tall’ uttered about a tree, but specific degrees vary a lot depending on the specific referent and context. Intuitively, the utterance “That’s big!” denotes quite different size degrees meanings, depending on whether it was uttered in reference to a flower or in reference to a truck, while both objects could potentially co-occur in the same context; it would be left to the listener to identify the correct referent. The aspect that goes unsaid and allows for this flexible use of the adjective across contexts is what the intended referent is big relative to. It seems plausible that these two objects could be compared to different things: it is more likely the case that the flower is big e.g. for this specific kind of flowers or relative to other flowers around it, whereas the truck is probably rather being compared to other trucks. 
	However, speakers rarely explicitly state this comparison class - the set of entities the target is compared against (Solt, 2009), and it is left to the addressee to establish the relevant comparison set. Listeners feature vast general knowledge and experience about the world helping them interpret context-sensitive language (e.g. Tessler et al. 2017), but what additional linguistic features might listeners attend to? In particular, how do listeners establish a comparison class in order to interpret a gradable adjective, given infinitely many a priori plausible options for the comparison class? 
	
