\documentclass[12pt, twoside]{report}
\usepackage[utf8]{inputenc}
\usepackage{graphicx}
\usepackage[a4paper,width=150mm,top=25mm,bottom=25mm,bindingoffset=6mm]{geometry}
%\usepackage{biblatex}
\usepackage{setspace}
\usepackage[english]{babel} 
%\usepackage{natbib}
\usepackage{csquotes}
\usepackage[style=authoryear, backend=biber]{biblatex}
%\DeclareLanguageMapping{american}{american-apa}
%\usepackage{apacite}
\addbibresource{references.bib}
%\bibliographystyle{plainnat}
%\setcitestyle{authoryear,open={(},close={)}}

\graphicspath{ {images/} }

\linespread{1.25}


\begin{document}
\begin{titlepage}
	\begin{center}
		\vspace*{1cm}
		\Huge
		\textbf{Inferring Comparison Classes of Gradable Adjectives\\} 
		\vspace{0.5cm}
		\Large
		\textbf{The Role of Informational Goals and Sentence Structure}
		
		\vspace{1cm}
		
	
		\textbf{By \\ Polina Tsvilodub}
		
		\vspace{1cm}
		\small
		Submitted in partial fulfilment of the requirements for the degree of \\
		Bachelor of Science in Cognitive Science \\ to the \\
		Institute of Cognitive Science at the Osnabrück University\\
		September, 28th 2020
		
	%	\vfill
		\vspace{3cm}
		Thesis Supervisor:\\ Prof. Dr. Michael Franke, Institute of Cognitive Science, Osnabrück University \\
		\vspace{0.5cm}
		Thesis Supervisor:\\ Dr. Michael Henry Tessler, Postdoctoral Associate, Department of Brain and\\ Cognitive Sciences, Massachusetts Institute of Technology  
		\vfill 
		\includegraphics[width=0.6\textwidth]{unilogo.jpg}
		
	\end{center}
\end{titlepage}

\chapter*{Abstract}
Understanding gradable adjectives like “big” requires making reference to a so-called comparison class - a set of objects the referent is implicitly compared to. For example, the utterance ‘“That Great Dane is big” could mean “That Great Dane is big compared to dogs in general” or “That Great Dane is big compared to other Great Danes”; yet the comparison class is rarely stated explicitly. So how do listeners establish the comparison class, given multiple a priori reasonable options?
Research on gradable adjectives has focused on the representation and integration of comparison classes into compositional semantics, but little is known about how human listeners decide upon a comparison class. 
This work takes a functional perspective on comparison class inference, guided by informational goals that speakers pursue when producing an utterance with a gradable adjective, and how listeners expect these goals to be achieved syntactically. For instance, given simple “Subject Predicate” sentences listeners expect that the subject aids reference (i.e., identifies the target), whereas the predicate accomplishes predication (i.e. asserts a property of the subject). Therefore, the role of the noun for comparison class determination  in a simple “Subject Predicate” sentence depends on its syntactic position. A noun appearing in the predicate is more likely to be intended to constrain the comparison class, whereas a noun in the subject can be explained away as intended for reference, leaving comparison class inference to other pragmatic reasoning. 
Converging evidence from four behavioural experiments supporting this proposal is presented alongside a novel formalisation of the inferential account in a qualitative computational model within the Rational Speech Act framework. This work contributes to the body of research on gradable adjectives, and provides a case study of  context-dependent language, emphasizing the complexity of the relation between the form and the meaning of linguistic expressions. 


\chapter*{Acknowledgements}
I want to thank...

\tableofcontents
\listoffigures
\listoftables

\chapter{Introduction}
The meaning of natural language expressions heavily depends on the context in which these expressions are used, but speakers rarely explicitly outline which aspects of the context are relevant for their interpretation. 

This issue is clearly illustrated by utterances involving gradable adjectives like \textit{big, small, tall, expensive} etc. These adjectives are typically taken to describe a \emph{degree} to which an object possesses some property, e.g., the degree of bigness (i.e., size) for the adjective ``big", but specific degrees a speaker intends to convey vary a lot depending on the particular referent and context. Intuitively, the utterance “That’s big!” denotes quite different size degrees, depending on whether it was uttered in reference to a flower or in reference to a house, while both objects could potentially co-occur in the same perceptual context; given this utterance, it is left to the listener to identify the correct referent and size degree. The aspect that goes unsaid and that allows for the flexible use of the adjective ``big" across referents and contexts is \textit{what the intended referent is big relative to}. Humans easily infer that two objects might be compared to different things: for instance, it is more likely that the flower in the previous example is big for its kind of flowers or relative to other flowers around it, whereas the house is probably rather being compared to other houses in the neighborhood. 

However, speakers rarely explicitly state a referent's comparison class---the set of entities the target is compared against---and it is left to the addressee to establish the relevant comparison class \parencite{Solt2009}. Listeners have vast general knowledge and experience about the world helping them interpret context-sensitive language \parencite{tessler2017warm}, but it is unclear what additional \emph{linguistic} features listeners might attend to when resolving context-sensitive ambiguity. 
In particular, previous work has left open the question how listeners establish a comparison class in order to interpret a gradable adjective, given infinitely many a priori conceivable options for the comparison class.

This work aims to fill that gap by investigating the role of syntactic structure for sentences containing gradable adjectives, and suggests that syntax provides a cue to contextually relevant aspects for adjective interpretation, which listeners integrate with other cues like perceptual context and world knowledge.\footnote{This thesis summarizes and extends the work by Tessler, Tsvilodub, Snedeker and Levy published in \textcite{tessler2020}, that appeared in the \textit{Proceedings of the 42nd Annual Meeting of the Cognitive Science Society}.} In particular, we hypothesize that syntactic structure reflects informational goals interlocutors strive to achieve; listeners reason about these goals pragmatically when inferring the comparison class of gradable adjectives. Focusing on the informational goals of \textit{reference} and \textit{predication}, this work presents a novel \textbf{reference-predication trade-off hypothesis} of comparison class inference, contributing to the body of research on gradable adjectives and providing a case study for the relationship between linguistic form and meaning. Evidence from four behavioral experiments is provided in support of this functional hypothesis, as well as a Bayesian model of gradable adjective interpretation, which shows that sophisticated pragmatic reasoning about syntactic structure can be captured using the generic probabilistic \emph{Rational Speech Act} framework \parencite{goodman2016}. 
	


\chapter{Understanding Gradable Adjectives}
Intuitively, the meaning of the positive form of a gradable adjective, for instance ‘big’, can be described as something like “big refers to the size of an object, and the size of that object that is described as big must be at least X, such that it counts as big, relative to some standard of comparison”. This means, gradable adjectives convey a feature, like size, and the degree to which the referent possesses this feature must exceed some threshold X for the referent to be felicitously described by the respective gradable adjective \parencite{Kennedy2007}. At the same time this threshold X can vary across contexts or categories: the minimal size of a flower that counts as big is quite different from the minimal size of a (real-life) truck that counts as big. Moreover, this threshold can vary within categories: the minimal size of a big sunflower is different from the minimal size of a big daisy, although both belong to the basic-level category flowers. Hence, this threshold X is heavily influenced by the set relative to which the object is compared - namely the comparison class. \\
That is, gradable adjective are vague - their meaning is subject to contextual variability, and to other characteristic features of vagueness: there exist so-called borderline cases, and these adjectives give rise to the Sorites paradox \parencite{Kennedy2007}. Specifically, even when a comparison class is set, there are cases where it is unclear whether an object counts as e.g. ‘expensive’: while a cup of coffee for \$1 is clearly cheap, and a cup for \$5 is clearly expensive, it might be difficult to say whether a  \$3.75 coffee is expensive or not - this is a borderline case. Using the same example, the Sorites paradox can be illustrated for gradable adjectives as follows: 
\begin{quotation}
P1: A \$5 cup of coffee is expensive (for a cup of coffee). 

P2: Any cup of coffee that costs 1 cent less than an expensive one is expensive (for a cup of coffee). 

C: Therefore, any free cup of coffee is expensive. 
\end{quotation}

It is the vague nature of gradable adjectives that makes it difficult to pinpoint why exactly people accept the premises so easily, and although the argument seems valid, the conclusion is clearly false (\textcite{Kennedy2007}, see for more details)
While investigating these important properties in greater detail is outside of the scope of this work, they exemplify here that representing this kind of implicit comparison to a threshold X which occurs in the positive form of gradable adjectives, while accounting for the existence of characteristics like borderline cases and eliciting the Sorites paradox is a difficult realm, which has posed difficulties to semantic theories. 

The following sections review state-of-art representations of gradable adjective semantics and the role of comparison classes therein. Then, a deeper analysis of comparison classes and prior related theoretical and experimental work is presented. 

\section{Representation of gradable adjectives}
Currently standard theories of gradable adjectives converge in representing gradable adjectives as a function mapping their argument - the referent - to a degree on an ordered scale representing some feature (e.g., ‘big’ and ‘small’ represent size),  utilizing degree morphology \parencite{Kennedy2007}. Degree morphology for positive forms of relative adjectives is informed by their comparative form, where the degree of a feature of the referent is explicitly compared to another degree of the same feature, and this comparison is overtly realised by a degree morpheme -er. For instance, in the comparative sentence ‘Bob is taller than Alice’ Bob’s height is explicitly compared to Alice’s height, expressed by the morpheme ‘-er’ appended to tall. 
By contrast, unmodified positive forms of relative adjectives which are the focus of this work don’t have an overt degree morpheme specifying the comparison; in the currently widely accepted approach (reviewed by \cite{Kennedy2007}) a phonologically silent null degree morpheme pos is introduced for this purpose. 
The morpheme pos takes the adjective as an argument and returns a standard of comparison - the context-dependent threshold (X). In \cite{Kennedy2007} the comparison class is assumed to be a property of the adjective, potentially restricting the domain of entities it applies to - an assumption discussed further in section 2.2. Formally, pos denotes the following:
%[[Deg pos λg λx. g(x)]]  = λg.λx: g(x)> s(λx: g(x) ) 
$$[[Deg_{pos} \lambda g \lambda x. g(x)  ]] = \lambda g. \lambda x: g(x) > s(\lambda x: g(x))$$

In other words, the degree to which the referent x possesses the property denoted by the adjective g must exceed some threshold, provided by s(g), where s is “a context-sensitive function that chooses a standard of comparison in such a way as to ensure that the objects that the positive form [of the adjective] is true of ‘stand out’ in the context of utterance, relative to the kind of measurement that the adjective [i.e. g] encodes” (\cite{Kennedy2007}, p.X). The contextually relevant aspects providing the threshold can be summarised as the comparison class of the adjective. 
For example, the expression ‘big dog’ is true if the size of a target dog exceeds some size-threshold, set by the comparison class. Depending on the context and the comparison class this threshold might vary: the minimal size the dog has to have in order to be described as ‘big’ is different if the dog is a toy dog, and the comparison class are other toys, than for a dog that is a Great Dane and the comparison class is other Great Danes.

Alternative to the degree-semantics framework, delineation-based formalizations of gradable adjectives treat them as unary predicates, forming partial functions depending on contextually provided comparison classes \parencite{Klein1980}. Such an approach removes degree representations from the semantics, although degrees arguably are an indispensable part of the meaning of gradable adjective \parencite{Solt2009}. 
The general issue of outlined semantics of gradable adjectives is that they assume the relevant comparison class to be supplied contextually, omitting specifying what exactly the comparison class is or how it is determined. 
While this work assumes a degree-based formalisation, it should be noted that alternative approaches also rely on the notion of contextually appropriate comparison classes, making the question addressed in this work as to how exactly comparison classes are determined a relevant one across different semantic representations.

\section{Understanding Comparison Classes}
Comparison classes can be understood as sets of entities, or reference frames the referent is compared against (\cite{Bierwisch1989}; \cite{Solt2009}; \cite{Klein1980}). In the outlined examples they were assumed to be sets of physical objects like dogs or toys. But comparison classes need not be comprised of individuals or objects, they can also comprise events or situations: In the utterance “The store is crowded for a Tuesday” the fullness of a particular store is naturally compared to other Tuesdays than to other stores \parencite{Solt2009}. Crucial is that “the comparison class provides statistical information that serves to determine the thresholds [...., and] what is relevant is not only the central value but also some measure of the extend of dispersion of values corresponding to members of the comparison class” (\cite{Solt2009} p.193).  
This naturally leads to the question is how exactly the standard of comparison - the threshold X - is determined by a given comparison class. For instance, \cite{Cresswell1976} suggested that the threshold X is the average of the relevant feature over the comparison class, but arguments have been laid against this idea, showing that these thresholds do not seem to comprise a single point on the degree scale, but should rather be represented as comprising a range of values (\cite{Kennedy2007}; \cite{vonStechow1984}). 
\cite{Solt2009} proposed that this range is computed as….

Comparison classes can be overtly expressed using prepositional for-phrases, for instance, as in “That Great Dane is big for a dog” or in “That shirt is big for you”. In the first example, additionally to expressing the comparison class, the for-phrase acts as a presupposition trigger, implying that the Great Dane is also a dog. Notably, this is not the case for the second utterance. 
\cite{Kennedy2007} suggested that for-phrases introduce a domain restriction on the gradable adjective via direct composition, hence being an argument of the adjective. That is, the comparison class restricts the domain of entities the adjective applies to. But this approach has difficulties accounting for cases when it is not the subject of the sentence that combines with the gradable adjective, or when adjectives appear in what has been labeled by \cite{ebeling1994children} as functional uses, e.g., “That short is big for you” (cf. Solt, 2011). An alternative is to interpret for-phrases in relation to the pos-morpheme, as marking its scope, similar to the relation between than-phrases and the comparative morpheme -er. In order to account for their presupposition-triggering behavior, the pos-morpheme is then assumed to take a comparison class as an argument, which by presupposition the referent is a member of (Solt, 2011). 
Formally: ….
This view follows the proposal by \cite{bartsch1973}, wherein the comparison class is an argument of a function computing the standard of comparison, whatever the nature of this function may be.   
Overt for-phrases may provide this argument in cases like “John is tall for a gymnast”, but for cases like “Sara reads difficult books for an 8-year-old” \cite{Solt2009} incidentally assumed ‘books’ to be the basis of the comparison class, rather focusing on the representation of the presupposition triggered by the for-phrase. In cases where no for-phrase is used, it is assumed that the argument of pos is a contextually appropriate implicit comparison class, e.g., supplied by the nominal modified by the adjective. Many assume this to be universally the case (cf. \cite{Cresswell1976}; \cite{Kamp1975}; \cite{Heim2000}), while \cite{Solt2009} restricts this mechanism to cases where the pos-morpheme is interpreted without raising. This leaves open the origin of the comparison class argument in sentences where the adjective appears predicatively, without a for-phrase - a question focused on in sections 2.3ff. . 
Finally, another approach to comparison class representation proposes that they “restrict binary relations, and these binary relations form the basis for the construction of [degree] scales [..., which] serve to relativize the calculation of a standard” of comparison (\cite{Bale2011}, p. 170). This proposal is based on deriving scales described by gradable adjectives from quasi-orders, i.e., those binary relations, for instance by creating so-called equivalence classes (sets of objects with equivalent degrees on that scale), which then are ordered based on the original quasi-ordering, and finally by defining a measure function via mapping each element onto its equivalence class in the scale \parencite{Bale2011}. The comparison class then restricts the quasi-order before the formation of the scale, restricting the quasi-orders to “ordered pairs consisting only of members of the comparison class”, such that the scale only describes degrees of members of the comparison class (\cite{Bale2011}, p. 178). This structure is then passed as an argument to some function returning the standard of comparison, analogous to approaches described above. One feature of this approach is the possibility to introduce a scale for gradable adjectives which are not inherently connected to some metric scale, e.g., for adjectives like beautiful or interesting \parencite{Bale2011}.   
This work focuses on the determination of the relevant comparison class, more so than on a specific semantic representation, so no commitment to a specific compositional approach shall be made here. 

Furthermore, comparison classes have been addressed from a developmental and psychological perspective. \cite{ebeling1994children} distinguish three prominent uses of gradable adjectives and comparison classes, respectively, which can be loosely related to distinct linguistic uses of those adjectives, namely occurrences of adjectives where the comparison class is supplied normatively, perceptually or functionally. 
Normative comparison class uses are based on a mental representation of the referent, for example it can comprise general world knowledge about the kind of things the referent belongs to. 
Perceptual comparison class occurrences are based on other objects of the same type as the referent physically co-present at the moment of utterance \parencite{ebeling1994children}. But the notion of perceptual comparison classes can naturally be extended to incorporate perceptually co-present objects in general (e.g. ‘That’s a big elephant’ could be said in a zoo intending to compare the elephant to other kinds of animals).
Finally, functional comparison classes are established with respect to the intended use of the referent (\cite{ebeling1994children}; \cite{sera1987}). 
While functional comparison classes are an exception in that they are very often stated overtly via a prepositional for-phrase (e.g. in ‘The shoes are too heavy for running’), both conceptual and perceptual comparison classes often remain implicit, left to the listener to infer from their world knowledge or relevant contextual aspects. This underspecification and context-dependence makes switching to these covert comparison classes form overt ones more difficult for small children, indicating the complex nature of gradable adjective interpretation \parencite{ebeling1994children}.

From a functional perspective, it can be noted that comparison classes comprising more general categories yield a wider distribution of property values than more category-specific comparison classes, yielding the latter generally more informative with respect to the meaning of the adjective. Hence, one could expect listeners assuming informative speakers to generally infer more informative - i.e., more restricted - comparison classes \parencite{tessler2017warm}.  However, this expectation might also trade-off with other pragmatic factors like typicality considerations (...) or a basic-level bias (discussed below in greater detail). 

While the notion of gradable adjectives as interpreted in reference to a comparison class has a long tradition (e.g., \cite{bartsch1973}; \cite{Bierwisch1989}), most of research in this domain has eschewed the question how exactly the relevant comparison class is identified, instead focusing on gradable adjective representation, once the respective comparison class is set (\cite{Kennedy2007}; \cite{Solt2009}; Kennedy, 2007; \cite{Kamp1975}).

\section{Compositional Accounts of Gradable Adjective Interpretation}
One line of prior work on how comparison classes might be determined approaches this question from a purely compositional perspective. In particular, the noun the adjective combines with is said to be at least a very salient contextual cue towards the comparison class (Kamp, 1975). 
Simple compositional accounts propose that the noun syntactically modified by the adjective necessarily stipulates the comparison class, such that ‘small watch’ resolves to ‘the watch is small for a watch’ (Kamp, 1975; Cresswell, 1976).
Yet, a lot of examples have been laid against such a simple mapping of the modified noun to the comparison class: intuitively, ‘John is a rich Fortune 500 CEO’ doesn’t mean that he is rich for a Fortune 500 CEO; ‘Kyle’s car is an expensive BMW’ doesn’t mean that his car is expensive relative to other BMWs (Kennedy, 2007). Furthermore, such a theory does not account for the interpretation of adjectives occurring predicatively, whereas a lot of adjectives - gradable adjectives in particular - can occur in either position (for example, attributively: ‘That’s a big dog’; or predicatively : ‘That dog is big’; cf. \cite{mcnally2008}; \cite{hofherr2010adjectives}).
The simplest generalisation of these compositional accounts to predicative adjectives would posit that the noun of the sentence generally sets the comparison class, such that the utterance “That Great Dane is big” would be taken to mean “That Great Dane is big for a Great Dane” (Tessler et al., to appear).
The exact syntactic relation between the two kinds of adjectives is widely discussed (e.g., \cite{Cresswell1976}; \cite{Kamp1975, hofherr2010adjectives}); however, the different potential formalisations would not affect the core of the presented proposal, since gradable adjective interpretation is approached from the perspective of the noun position rather than the adjective position, wherein the syntax only provides cues to high-level informational goals, as opposed to stipulating specific low-level syntactic relations.

Another distinction along the lines of adjective position potentially relevant for comparison class determination is the difference between intersective and non-intersective (or subsective) adjective readings \parencite{kennedy2012, hofherr2010adjectives}. 
Intersective adjective interpretations emerge when the target is interpreted as a member of the intersection of two sets: the one denoted by the noun and the one denoted by the adjective \parencite{kennedy2012}. For example, the sentence ‘Bobby is an expensive dog’ implies that Bobby is both expensive (in general) and that he is a dog. 
In contrast, subsective interpretations emerge when the referent is interpreted as a member of a subset of the set denoted by the noun, returned by the adjective combining with the noun: For example, the sentence “John is a skillful surgeon” implies that he is a surgeon, but not necessarily that he is generally skillful - it only implies that he is skillful as a surgeon \parencite{kennedy2012}. Nonsubsective adjectives like ‘former’ can be disregarded for our purposes. 
Although it is argued in the literature that specifically prenominal attributive adjectives give rise to ambiguity between the two readings (cf. ‘Olga is a beautiful dancer’; \cite{hofherr2010adjectives}), it seems plausible to a priori treat gradable adjectives occuring in either position (attributive: ‘That’s a big dog’, or predicative: ‘That dog is big’) as eliciting intersective interpretations (i.e. both amount to the reading ‘the referent is both a dog and is generally big’), therefore leaving the comparison class underspecified. Positing a subsective reading for either sentence frame (i.e. ‘the referent is big for a dog, but not generally big’) would amount to the simple syntactic hypothesis wherein the noun sets the compison class, which intuitively can not hold in general (especially given examples like “John is a rich Fortune 500 CEO”: positing a subsective reading would translate to the sentence “John is rich for a Fortune 500 CEO, but not rich in general”, which intuitively isn’t correct). The ambiguity mentioned above would then rather be ascribed to the comparison class ambiguity, than to the ambiguity of the denotation set. 

Syntactic accounts outlined above stipulate that the meaning of an utterance involving gradable adjectives is fully specified by its words. Yet it was shown that several other pragmatic components like context of the utterance and listeners’ world knowledge seem to come together when gradable adjectives are interpreted. Psycholinguistic studies investigated the role of these pragmatic factors for gradable adjectives and comparison class determination. 

\section{Pragmatic Accounts of Gradable Adjective Interpretation}
Several visual world studies addressed the role of context for relative adjective interpretation, specifically focusing on manipulating the distractor set and the feature degrees of the distractors in the context in which a target described by an adjective was presented (e.g., \cite{sedivy1999, Aparicio2015}). 
They considered prenominally modified nouns in predicative position for the goal of reference resolution, showing that targets were identified faster when the context supported restrictive interpretation of the predicate (Sedivy et al., 1999; Aparicio, Xiang, Kennedy, 2015). That is, then context included another object of the same kind as the target, but with a different degree of the feature described by the adjective, such that the adjective was interpreted as helping ‘restrict’ the pair of objects to the target . By contrast, this effect was not shown in contexts including distractors of other categories than the target \parencite{Aparicio2015}. 
Notably, both studies assume that comparison classes of gradable adjectives were supplied contextually, and more so when the context was more homogenous (i.e. included several objects of the same type) (cf. \cite{Aparicio2015}), showing the context-dependent nature of comparison class determination. 

Another recent study addressed the role of world knowledge for comparison class inference \parencite{tessler2017warm}. The authors showed that listeners flexibly adjust comparison classes of gradable adjectives based on their world knowledge, when encountering simple utterances like “He’s tall” said of targets about which listeners typically have strong expectations regarding the degree of the feature described by the adjective. They showed that listeners are more likely to infer that an utterance like “He’s tall” said of a basketball player means “He’s tall for a person” (i.e. relative to a general, basic-level category), whereas the utterance “He’s short” rather means “He’s short for a basketball player” (i.e. relative to the target’s subordinate category). This pattern was clearly shown for targets of those categories which exhibit a rather high or low degree of the feature (e.g. basketball players, whose height is generally quite large; and jockeys, whose height is generally rather low) \parencite{tessler2017warm}. 
The present studies build on this experimental paradigm, making use of listeners’ expectation about such categories highly salient with respect to some feature. When adjectives describing this certain feature scale are attributed of those categories, the basic-level comparison class is a priori more likely to provide a felicitous expression than the subordinate comparison class. 
However, the study by \cite{tessler2017warm} only considered simple utterances, appearing without much context or a noun. 


\chapter{A Fucntional Perspective on Comprison Class Inference}
This section aims to integrate both the role of the noun in the sentence as well as the role of pragmatic cues like perceptual context and world knowledge for relative adjective interpretation, presenting the \textbf{reference-predication trade-off hypothesis} of comparison class inference. 

Specifically, the issue of comparison class determination is approached from a functional perspective, based on the question what \emph{informational goals} speakers might pursue when producing an utterance containing an adjective, and how these goals might influence listeners’ comparison class inferences \parencite{tessler2020}.
The proposed approach is an inferential account of comparison class determination, informed by the idea of recursive social reasoning mechanisms, applied to rational language use in Gricean tradition: Speakers have certain informational goals which guide how they craft their utterance in order to facilitate message interpretation with respect to these particular goals for a listener \parencite{goodman2016}. Listeners, in turn, infer the most likely state of the world - that is, in case of gradable adjectives, the most likely comparison class - in light of those speaker goals. 

In particular, in contrast to cases considered in eye-tracking studies described in chapter \ref{chapter02},  when using adjectives speakers might also primarily intend to convey a property of a target referent. In order to communicate that property of a referent, speakers must achieve at least two informational goals: \textit{reference} - identifying the right target - and \textit{predication} - attributing a property of the target, which in case of relative gradable adjectives amounts to communicating the specific degree of the feature denoted by the adjective \parencite{Reboul2001, Kennedy2007}.  
For these two informational goals, it is reasonable to posit that listeners generally expect the subject to be sufficient in order to establish reference - independent of the predicate asserted to hold of the subject \parencite{Reboul2001, syrett2010meaning, searle1969speech}. Cooperative speakers then aim to satisfy this general expectation.

This tendency is particularly strong for sentences with subjects containing referential expressions like definite descriptions, pronouns or deictics (cf. section \ref{2.4.}). Furthermore, it might be based on general information structural reasons: In order to predicate a property of a target, this target must be clear \parencite{searle1969speech, krifka2008basic}. Therefore, the subject also tends to convey the \textit{topic} of an utterance - that is, "the entity under which the information from the comment constituent should be stored" \parencite[p.X]{krifka2008basic}; while the predicate tends to convey the \textit{comment},  i.e., potentially new information about that entity \parencite{krifka2008basic, chafe1976givenness, Reboul2001}. A further heuristic distinction associated with the subject-predicate contrast comes from linguistic packaging literature, wherein the predicate is assumed to convey the \textit{main news} (as opposed to \textit{secondary information}), and also potentially \textit{new information}, while the subject might convey secondary information which is already \textit{known} \parencite{kaiser2020}. 

Note, however, that there are exceptions to many of these tendencies: for instance, for the sentence “The boss fired the worker because he was a convinced communist” the pronominal \emph{he} can be resolved not only after applying the predicate, but also only taking into account the context - \emph{he} can either refer to the boss or to the worker \parencite{Reboul2001}. \textcite{krifka2008basic} also points out that the topic, and hence the subject,  doesn't necessarily convey known information.
Yet we posit that these structural expectations are a general enough heuristic holding in many contexts.

These expectations have implications for comparison classes of gradable adjectives insofar as to speakers have the liberty to choose from truth-conditionally similar sentence options to communicate the same message. For example, in order to tell some 4-year-old kids on a playground in winter that they built a big snowman, a speaker has the liberty to say “That’s big!” pointing at the snowman, “That snowman you built is big!” or “You built a big snowman!”, among many other options. Consequently, the choice of a particular sentence over other equivalent options might respond to particular informational - communicative needs. 

From this perspective, the influence of the noun on the comparison class in a simple \textit{Subject Predicate} sentence depends on its position in the sentence. If the noun appears in the predicate of the sentence (e.g., in “That’s a big Great Dane”), it can naturally be explained as produced by a speaker intending to constrain the comparison class, by packaging the noun along with the adjective as the most important information. By contrast, if the noun appears in the subject of the sentence (e.g., in “That Great Dane is big”), it can potentially be \emph{explained away} as produced by a speaker who intends it to support reference (especially via combining it with the deictic ‘that’), and hence the noun is a weaker cue towards the comparison class. The comparison class inference is then guided by other pragmatic cues like world knowledge or perceptual context (Fig. \ref{model-cartoon}).

\begin{figure*}[t]
	\begin{center}
		\includegraphics[width=0.9\linewidth]{ref-pred-cartoon-w-subscripts2.pdf}
	\end{center}
	\caption{Cartoon of the inferential account for comparison class determination. The noun (Great Dane) in a sentence can be employed either for the goal of reference (green) or predication (purple), shown in the case when this distinction is made via the syntactic position of the noun (subject S~vs.~predicate P). When the noun is used for reference (top), a listener is left with uncertainty about what to use as the comparison class (dogs or Great Danes) and integrates their world knowledge and the physical context to make this inference.  When the noun is used for predication (bottom), the listener should have less uncertainty about the comparison class: The comparison class is stipulated by the noun.}
	\label{model-cartoon}
\end{figure*}
 
Hence, the utility of the noun as constraining the comparison class is the result of a trade-off between its utility in reference and predication, such that comparison class inference is guided by integrating syntactic with other contextual cues.

\section{Understanding Reference and Predication}
\label{3.1.}
This reference-predication trade-off hypothesis focuses on two basic informational goals, reference and predication, which have been discussed in a great deal of work in semantics, pragmatics and philosophy of language \parencite{michaelson2019, Reboul2001}.
 
\textcite{searle1969speech} conceptualizes both reference and predication as particular kinds of propositional acts, defining conditions to be fulfilled in order to accomplish them. Of particular importance for accomplishing reference is that the expression intended for reference isolates the target referent for the listener \parencite{searle1969speech}. Studies have shown that speakers are aware of this requirement, and being sensitive to contextual variability, adjust the informativity of their referential expression correspondingly, such that this requirement is satisfied \parencite[e.g.,][]{graf2016animal}. In particular, definite descriptions which prenominal adjectives might be a part of have been the focus of a lot of work on reference, converging on the claim that a singular determiner phrase of the form \emph{the $\phi$} triggers two presuppositions: the \textit{existence} presupposition (i.e., that there is an object satisfying the description $\phi$), and the \textit{uniqueness} presupposition (i.e., that such an object is uniquely identifiable) \parencite{syrett2010meaning, michaelson2019}. These same presuppositions generally also hold for pronouns and demonstratives, but do not for indefinite descriptions of the form \textit{a $\phi$} \parencite{braun2017, Reboul2001}. Therefore, our experimental operationalization focusing on predication employs gradable adjectives in indefinite descriptions (s. section \ref{3.2.})

The goal of predication builds upon reference, in that one of the requirements for accomplishing predication is that the same sentence contains a reference to the intended target of predication \parencite{searle1969speech, Reboul2001}. Specifically for relative adjectives, predication is tantamount to communicating a particular property degree, and therefore supplying a felicitous comparison class, for the referent under discussion. Accomplishment of the goal of predication is often roughly equated with the syntactic predicate, which notably might consist of a bare predicative gradable adjective, introduced with a copula. Therefore, one might hypothesize that the noun cannot be the only cue to the comparison class, since predication might be accomplished by a bare adjective.

This review does not attempt to resolve the debate on how exaclty reference and predication might be accomplished. But of particular importance for this work is the flexibility of nouns with respect to both informational goals: combining with the deictic ‘that’, the noun can accomplish reference; but being part of a non-referential expression (e.g., an indefinite description), the noun can contribute to predication \parencite{Reboul2001}. 

The focus of this work are these two relatively basic informational goals, but clearly there are other communicative uses of adjectives. For example, \textcite{barker2002dynamics} distinguishes between \textit{descriptive} and \textit{meta-linguistic} uses of vague adjectives. The former refers to what so far has been considered \emph{predication} applied to relative adjectives, while the latter refers to giving 'guidance concerning what the prevailing relevant standard' of comparison is for the adjective under discussion \parencite[p. 2]{barker2002dynamics}. That is, the goal in this case is to teach the appropriate use of the vague adjective, given a particular property value in context. Another related goal of adjective use might be conveying a subjective opinion about a property \parencite{kaiser2020}. Interestingly, gradable adjectives have been shown to differ in the degree of subjective content they might convey \parencite{scontras2017subjectivity}. Further investigation of these communicative goals and their relation to reference and predication is left open to future research.

The discussed properties of reference and predication lead to the particular experimental operationalisation of the reference-predication trade-off hypothesis, described in the next section. 

\section{Experimental Operationalization}
\label{3.2.}
In present studies, the flexibility of nouns to contribute to either informational goal leads to the operationalization of the reference-predication trade-off hypothesis via a syntactic manipulation, wherein the noun (N) which combines with the gradable adjective (ADJ) appears either in the subject or in the predicate of a sentence. Experiments 1-3 employ sentences including only one critical noun N \parencite{tessler2020}:
\begin{quotation}
	\textit{Subject N}: That N is ADJ. 
	
	\textit{Predicate N}: That's a ADJ N.
\end{quotation}
Experiment 4 focuses on the critical noun N1 syntactically modified by the adjective, which then appears either in the subject or in the predicate of an utterance: 
\begin{quotation}
	\textit{Subject N}: That ADJ N1 is a N2. 
	 
	\textit{Predicate N}: That N2 is a ADJ N1.
\end{quotation} 
Given the referential presupposition of the deictic 'that', subject nouns should be taken as establishing reference.  For the predicate noun condition, reference should be taken as being established by the bare deictic or the second noun N2, respectively. Given the presuppositional nature of definite descriptions, the predicate N conditions were chosen to include an indefinite description, such that the predicate may apply to several members in context and referential pressure be shifted to the subject of the utterance. Furthermore, in the experimental set-up the referent described by critical sentences was perceptually salient, and the task did not involve direct reference resolution, such that referential pressure was generally lower than in experiments described in section \ref{2.4.}.
\pt{discuss in chapter 6 that deconfounding definiteness from syntactic manipulation should be addresses in future research; keep it maximally symmetric in E1-3; tentative predictions for E4: same distinction for "A prize-winner is a big great dane" vs "A big great dane is a prize-winner"; infelicitous presuppositions for both parts being definite; also discuss connection to plural / generics / predagogical language;}

The critical question addressed by this manipulation is how speakers and listeners treat these syntactic frames, asserting the ADJ of referents for whom they are felicitous given one comparison class, but not another (e.g., a \emph{normal-sized} Great Dane can felicitously be described as ‘big’ given the comparison class ‘dogs’, but not ‘Great Danes’). 

The reference-predication trade-off hypothesis predicts that nouns that are more likely to establish reference are less likely to constrain the comparison class. Therefore, when the noun appears in the subject of the utterance, it can be explained away as establishing reference, and hence is a weaker cue towards the comparison class, leaving it open to influences of world knowledge and perceptual context. 

 Conversely, when the noun is taken to contribute to predication, i.e., when it appears in the predicate of the sentence, it is more likely to constrain the comparison class. Therefore, this noun is rather expected to be consistent with the comparison class felicitous in order to describe a target: for instance, the basic-level category label would be more appropriate for setting the comparison class when describing a normal-sized Great Dane as ‘big’ than the subordinate category label. That is, the utterance 'That's a big dog' would be more appropriate than 'That's a big Great Dane' in order to describe a normal-sized Great Dane, because the subordinate category \emph{Great Danes} is generally a large-subordinate category compared to the basic-level category \emph{dogs}, but normal-sized representatives are not necessarily large compared to their subordinate category. 

Operationalisation of referential utility following reference-RSA. 

Note that although the differences in comparison class restriction are approached through the lense of this syntactic manipulation, the underlying communicative goals are the primary driving force in comparison class inference, to which the syntax is just a cue. 
There might well be other syntactic realisations of these informational goals (Reboul, 2001): The sentence “What is big is that Great Dane” seems appropriate in a context where generally big things are discussed; in this utterance reference is accomplished from the predicate, and because of this referential pressure, under the trade-off hypothesis the noun would not be expected to constrain the comparison class, although it appears in the predicate, supporting the view that the syntactic position of the noun is dissociable from the intended communicative goals.

To show that informational goals are primary for comparison class inference as opposed to specific syntactic properties of the adjectival phrase, experiment 4 focuses on manipulating the informational goal the noun is a cue to, while it is directly syntactically modified by the adjective. This manipulation allows to disentangle the effect of the noun position from the effect of syntactic modification of the noun, which are confounded in experiments 1-3. For example, critical sentences in experiment 4 are “That big Great Dane is a prize-winner” (subject-N) or “That prize-winner is a big Great Dane” (predicate-N). The trade-off hypothesis predicts that even directly modified nouns in the subject position contribute to reference, and thus should be less likely to constrain the comparison class, compared to nouns appearing in the predicate.

The next chapter presents results of four behavioural experiments exploring the reference-predication trade-off hypothesis, specifically investigating the use of the size adjectives ‘big’ and ‘small’. These two adjectives are chosen for practical reasons: size is a visually accessible feature, allowing for easy presentation and manipulation of the context in web-based experiments. Furthermore, humans usually have strong expectations about typical size distributions of different natural categories, from which the target referents were sampled for the experiments. Three distinct dependent measures were used to assess the influence of various cues on comparison class inference. This experimental data provides a comprehensive overview of pragmatic and syntactic effects on comparsion class inference. 

\chapter{Experiments}
The reference-predication trade-off hypothesis was tested in four behavioural web-based experiments employing different dependent measures. The crucial manipulation in all experiments was the varying position of the critical noun - it appeared either in the subject (e.g., “That N is ADJ” or “That ADJ N is N2”) or in the predicate (“That’s a ADJ N” or “That N2 is a ADJ N”) of the sentences presented in the experiments. These sentences described an object which appeared in visual context. 
These objects were sampled from five different basic-level categories (Rosch et al., 1976): dogs, birds, flowers, trees and fish. Within each basic-level category, at least two subordinate categories were chosen which exhibit a rather high or rather low amount of the feature described by the gradable adjectives under investigation - that is, those subordinate categories were chosen which people expect to be rather large or rather small representatives of their basic-level categories. For example, for the dog-category the large-subordinate category Great Danes and the small-subordinate category pugs were chosen. As shown by Tessler et al. (2017), when encountering representatives of such categories described by the adjective consistent with participants’ prior expectations about the degree of feature-under-discussion, people are a priori more likely to infer the basic-level comparison class than the subordinate comparison class. For example, when encountering the sentence “It’s big” said of a Great Dane (a large-subordinate category for the basic-level category dogs), humans are more likely to infer that the Great Dane is big relative to other dogs in general, than big relative to other Great Danes.  
Following the design of Tessler et al. (2017) allows to test the effect of syntactic position of the noun on how strong the noun is taken to constrain the comparison class: The reference-predication trade-off hypothesis predicts that nouns in the predicate position constrain the comparison class more strongly than in the subject position, such that a priori using the basic-level noun in predicate position is more felicitous in order to describe a normal-sized large-subordinate object (e.g. a Great Dane) than using a subordinate-label of the object in predicate position. Both nouns would be felicitous in the subject position. In all experiments, the referents were described by the adjective matching prior feature-degree expectations, i.e. Great Danes and sunflowers were always described as big, and pugs or daisies as small. Table 1 shows all the stimuli used in all experiments. 

The structure of all experiments was similar. First, participants completed a bot-check trial: Participants read a sentence where a named speaker asks a named listener: “It’s a beautiful day, isn’t it?”. The speaker and listener names are sampled from lists of ten most popular male and female English names. For example, the sentences say: “John says to Mary: “It’s a beautiful day, isn’t?”; Who is John talking to?”.  Participants are asked to fill-in in lowercase who the listener is talking to. Participants are provided feedback and have maximally 3 attempts to fill-in the correct name. They are only allowed to proceed, if they successfully complete the bot check. Then, participants read instructions and complete practice trials, before completing main trials. After the main trials, they complete a socio-demographic post-test questionnaire, where they are asked to indicate their native language and optionally further information. 
For all experiments, participants were recruited via the crowd-sourcing platform Amazon’s Mechanical Turk; only participants with IP addresses in the United States and work approval rating of at least 95% were permitted to participate. Participants were restrained from taking part in multiple experiments of this series.  

The first experiment (E1, Sentence Rating Experiment) was a sentence rating experiment, wherein participants had to rate two sentences which differed in the position of the noun and the specificity of the noun, as describing an object in context. 
In the second experiment (E2, Noun Production Experiment), participants had to fill-in the missing noun of a sentence describing the size of a referent in context. The position of the missing noun was varied. 
In the third experiment (E3, Comparison Class Inference Experiment), participants provided the inferred comparison classes via a free-production paraphrase, given sentences which varied by the noun and its position, as describing a referent in different contexts (Tessler et al., 2020). 
Finally, the fourth experiment (E4, Direct Modification Experiment) investigated a potential confound - the syntactic modification of the critical noun - and gathered inferred comparison classes in a paradigm akin to experiment three, but from sentences wherein the critical noun appearing in subject or in predicate was always syntactically modified by the adjective. 
All experimental materials can be found under https://github.com/polina-tsvilodub/refpred, the preregistrations can be found under ….. All experiments were realized using the \_magpie - framework %Ilieva, Ji, Rautenstrauch and Franke, https://magpie-ea.github.io/magpie-site/). 
The data from these experiments provides a suite of evidence for the role of world knowledge, context and syntactic cues on the interpretation of the gradable adjectives “big” and “small”.  

\section{Experiment 1: Sentence Rating Experiment}

Results: main, by-participant / by-item variation, different exclusion criteria 

The aim of the sentence rating experiment was to investigate whether participants prefer one syntactic frame over the other, given two truth-conditionally equivalent sentences, involving different nouns. The type of the noun and its syntactic position differed within-subjects.
First, participants completed two warm-up trials to familiarize themselves with the slider rating procedure. On one trial, participants read: “Imagine you see this basketball” above a picture of an orange basketball, and read below the question: “How well does each of the sentences describe it? (Please click on the slider to provide a rating)”. Two sentences appeared below: “The basketball is orange” and “The basketball is green”, to be rated on sliders ranging from “very bad” to “very well”. In the background, the ratings are mapped onto a scale ranging from 0 to 100. The slider is light gray, with a round handle appearing upon clicking on the slider track. The same sliders were used in the main trials. On the other trial, participants read: “Imagine you see this chair” above a picture of a purple chair. The sentences to be rated appearing below were: “The chair is yellow”, and “The chair is blue”. The order of the warm-up trials was randomized.    
Then, participants completed six main trials. Participants read “You and your friend see the following:” above a basic-level context picture (e.g. a group of dogs). The picture consists of six other members of the same basic-level category as the referent of the trial, including two other members of the same subordinate category as the referent. The six members consist of two members of a large-subordinate, a medium-sized subordinate, and a small-subordinate within the basic-level category each (e.g. the dog-context consists of two Great Danes, two poodles and two pugs). The context is used to set the overall reference comparison class for the targets. It also sets the visual reference frame.
Below, they read the sentence “You also see this SUB\_N”, where SUB\_N is the subordinate label of the target referent, which appears depicted below. The pictures depict referents a little smaller than members of the same subordinate category in the context, such that the felicitous comparison class is pushed towards the basic level category of the target.
Below, the question about the critical sentences appears: “How well does each of the sentences describe it? (Click on the slider to provide a rating)”. Then, the two critical sentences appear left of the sliders one below the other. The sliders range from “very bad” to “very well”. On every trial, in one of the sentences the noun appears in the subject (e.g. “That N is {big, small}”), in the other in the predicate position (“That’s a {big, small} N”). The order in which these syntactic conditions appear is randomized between-subjects. 
On half of the trials the noun is the basic-level target label (e.g. dog), on the other half it is the subordinate target label (e.g. Great Danes), balanced within-subjects. 
Participants see each of the six possible context once, and for each context, one of the two possible targets (large-subordinate vs. small-subordinate category representatives) is sampled, balanced within-participants. 
Example trial view.
The reference-predication trade-off hypothesis predicts that sentences with a basic-level noun in the predicate position should receive a higher rating than sentences with a subordinate noun in predicate position, but there should be no difference in the ratings of the sentences with a basic-level and  subordinate noun in the subject position.  

\subsection{Participants:}
113 participants were recruited and 33 were excluded for indicating a native language other than English, failing the practice trials or providing the same responses on every trial (see Appendix X). The experiment took about 5 minutes and participants were compensated \$0.80. If partial data was missing from a participant, available data was used for analyses. 
\subsection{Results:}
A Bayesian linear mixed-effects regression model was fit, predicting the sentence rating from the syntactic condition of the sentence (subject vs predicate noun), the noun type (basic-level vs. subordinate target label), their interaction and random  by-participant and by-target random intercepts and random effects of syntax, noun type and their interaction. An exploratory model including a main effect of syntactic condition order was also fit, revealing no effect of syntactic condition order [...], so the data was collapsed across the two conditions. Both predictors are deviation coded, coding both the subject-noun and the basic-level noun as 1 and the other levels as -1, respectively. 
Consistent with the predictions, participants substantially dispreferred sentences with a subordinate noun in the predicate compared to the subordinate position, but no effect of syntax was found for the basic-level nouns, as indicated by the syntax X noun-type interaction [...]. Additionally, an overall preference for basic-level nouns [...] and the subject-noun syntactic structure [...]. In an exploratory analysis including a predictor of target size (small-subordinate vs. large-subordinate category) revealed that …. . Furthermore, a relatively high by-subjects  and by-target variance revealed that …. .


\section{Experiment 2: Noun Production Experiment}    

classification of responses
results: main, by-participant / by-item, by-size  

The goal of the noun production experiment is to investigate whether in a free-production setting participants produce nouns of different categories, given different syntactic frames.  The noun of the critical sentences in the main trials appeared either in the subject position (e.g. in “That N is {big,small}“) or in the predicate position (e.g. in “That’s a {big,small} N”), manipulated between-subjects. 

In this experiment, participants completed two blocks, each consisting of three warm-up trials and three main trials. In the warm-up trials participants familiarize themselves with the subordinate categories used in the main trials. They see pictures of a member from a large-subordinate and a small-subordinate category within one of the basic-level categories used in the main trials (e.g. a Great Dane and a pug). Participants are prompted to provide labels for these pictures. Below they are prompted to provide a common label for both pictures (i.e. dogs). They are provided feedback for the labels and can proceed upon adjusting their labels to correct responses. The number of attempts participants needed until they filled-in the correct labels is recorded. In this experiment, four additional subordinate categories were used, which can be found in Table 1 marked with *. For each participant, six out of ten possible contexts are sampled. Three of these contexts and their corresponding targets appear in the first experimental block, and the other three in the second. The trial order within the warm-up block and the main block is randomized. 
On the main trials, participants read: “You see the following:” above a basic-level context picture, akin to the contexts used in experiment 1. Below, they read “You also see this one:” and see a picture of the target referent. Then they read: “You say to your friend:”, prompting them to fill-in the missing noun in the sentence: for the subject-noun condition, the template is “That\_\_ is {big, small}”, for the predicate-noun condition, the template to be completed is “That’s a {big, small} \_\_ “.  
The size of the target referent was balanced within-participants: on three trials, participants see referents from a small-subordinate category, and on three, they see referents from a large-subordinate category. For each context, participants see only one of the possible targets (e.g. the large or the small subordinate target).
Example trial view / example warm-up views
The reference-predication hypothesis predicts that participants should be less likely to produce subordinate target labels in the predicate compared to the subject position.

\subsection{Participants:}
242 participants were recruited, and 52 were recruited for indicating a native language other than English or for failing the warm-up trials. The exclusion criterion was taking more than four attempts on any warm-up trial to provide the correct label upon correction. The experiment took about 7 minutes and participants were compensated \$1.00. 
 
\subsection{Results:}
The responses provided by participants were categorized manually into basic-level or subordinate-level labels of the targets. X \% were superordinate referent labels (i.e. more general labels like animals) and were collapsed with basic-level labels. 16 (1.4\%) uncategorizable responses were excluded from analysis. 
A logistic generalized mixed-effects regression model was fit, regressing the response category (basic-level. vs subordinate target label) against the syntax of the sentence, random by-participant and by-referent intercepts and random by-referent slope effects of syntax. 
Consistent with predictions, a strong effect of syntactic position of the noun was found, indicating that participants were appreciably more likely to use basic-level labels in the predicative position (2.25 [0.74, 4.01]). 
Furthermore, different participants showed different sensitivity to the syntax, as indicated by… Additionally, by-referent variation was found, which could be attributed to differing namability of the targets. ….
An exploratory model including a main effect of referent size (large-subordinate vs. small-subordinate category) revealed that... 
 
 
\section{Experiment 3: Comparison Class Inference Experiment}
The two previous experiments support the reference-predication trade-off view, by showing that participants disprefer sentences like “That’s a big Great Dane” in order to describe a normal-sized Great Dane. The goal of this comparison class inference experiment was to measure comparison class inferences more directly, as influenced by the position of the critical noun in the sentence, the type of noun and the visual context of the sentence. All three factors are manipulated within-subjects.
When participants don’t have access to visually assessing the size of a referent and have to infer the comparison class from the sentence, they should be sensitive to linguistic cues like the sentence structure. According to the outlined hypothesis, they should be more likely to take the noun as a cue to the comparison class when the noun appeared in the predicate of that sentence, than when it appeared in the subject. When the noun appeared in the subject, comparison class inference can be driven by other pragmatic inference, e.g. from world knowledge and visual context. 
First, participants completed a comparison class paraphrase practice trial, akin to the paradigm employed in the main trials. Participants were told that on the main trials they will see a sentence containing a word that is relative, and their task will be to figure out what this word is relative to. They read an example task: “Speaker A: ‘The Empire State building is tall.’ What do you think speaker A meant?”. Below they saw a paraphrase template where they provided the inferred comparison class of the adjective tall: “The Empire State building is tall relative to other\_\_” (blank to be completed with the inferred comparison class). Participants were provided feedback on their response and had to correct it to one of the possible options among {buildings, skyscrapers, houses, constructions}. 
Then, participants completed two blocks consisting of labeling warm-up trials and main paraphrase trials. Three of the six basic-level categories used in this experiment (table 1) are sampled for the first block, with the respective subordinate category members appearing in the warm-up trials, the other three categories appear in the second block. These labeling warm-up trials are of the same kind as in Experiment 2. 
In this experiment, for the main trials there were basic-level and subordinate-level contexts for each possible referent. Basic-level contexts were identical to the contexts of respective categories in Experiment 1 and Experiment 2; the subordinate contexts consisted of six other representatives of the same subordinate category as the target referent. For example, the subordinate context for a Great Dane consisted of a picture of a group of six other Great Danes. Within each main trial block, there were six trials, wherein for each of the three sampled categories, one possible referent appeared in the corresponding basic-level context (e.g. for the category flowers, the sunflower appeared in basic-level flower context), and the other possible referent appeared in the corresponding subordinate context (i.e., then the daisy appeared in subordinate daisies-context). 
The referent was described by a critical sentence in which the noun could appear in the subject or in the predicate of the sentence. The noun could be either the basic-level (e.g. dog) or the subordinate label of the referent (e.g. Great Dane). Furthermore, a baseline condition with an anaphoric ‘one’ in the noun position was included, in order to measure the baseline influence of the visual context on comparison class inference (Goldberg and Michaelis, 2017). Crossing the visual context (basic vs. subordinate), the syntax (subject-N vs. predicate-N) and the possible nouns (basic vs. subordinate vs. ‘one’) results in a 2x2x3 design, yielding 12 unique conditions. Each participant saw each condition once during the total of 12 main trials.   
On main trials, participants read “You and your friend see the following:” above a context picture. Below, they read: “Your friend runs far ahead of you, and you see him in the distance:”. The illusion of distance was created contextually in order to disguise the perceptual size of the target referent and push participants towards inferring the size of the referent from the sentence, rather than perceptually. This illusion was supported by the picture appearing below, wherein the small target referent was depicted next to a small person (as compared to the context, i.e. appearing in distance). Below, participants read: “Your friend says:”, followed by the critical sentence. Participants were asked “What do you think your friend meant?”, followed by the paraphrase template “It is {big, small} relative to other \_\_”, blank to be completed with the inferred comparison class. 

\subsection{Participants:}
245 participants were recruited and 45 were excluded for indicating a native language other than English, or failing either the comparison class inference practice trial or the labeling warm-up trials more than four times upon correction. The experiment took about 9 minutes and participants were compensated \$1.20. 
\subsection{Results:}
by-target / by-participant variation
by-size?
Participants’ responses were manually classified into basic-level and subordinate comparison classes. X superordinate comparison classes were collapsed with the basic-level responses. 39 (1.6\%) uncategorizable responses were excluded from the analysis. 
A Bayesian generalized logistic mixed-effects regression model was used, regressing the response category against the syntactic condition (subject-N vs. predicate-N), the noun category (basic vs. subordinate vs. ‘one’), the context (basic vs. subordinate), their two-way and three-way interactions and maximal random effect structure appropriate for this experimental design (footnote formula). 
The results indicate that participants flexibly adjust the inferred comparison class according to many factors. First and foremost, a large effect of visual context going above and beyond other factors was found, providing evidence against a simple purely syntactically-oriented theory of gradable adjective interpretation; as indicated by the inferences drawn from the baseline condition anaphoric ‘one’. Furthermore, an effect of noun regardless of its position in the sentence was found: participants were more likely to infer basic-level comparison classes from basic-level nouns than from subordinate nouns. Notably, the subordinate comparison class was the minority response given a subordinate noun in the basic-level context, speaking against a modificational view of adjective comparison classes. Crucially, a credible syntax X noun interaction was found, supporting the reference-predication trade-off hypothesis: more subordinate comparison classes were inferred from subordinate nouns appearing in predicate position than in the subject position, compared to basic-level nouns. 
Exploratory model with the main effect of context; comparison to one -- subordinate noun driving the interaction

\section{Experiment 4: Direct Modification Experiment} 
In order to keep a simple and interpretable operationalization of the reference-predication distinction, a potential confound was introduced in Experiments 1-3. The position of the noun was perfectly confounded with whether the noun was syntactically modified by the adjective (predicate-N condition) or not (subject-N condition). However, the reference-predication trade-off view predicts that referential pressure takes off some weight from the noun used for reference and decreases its strength in constraining the comparison class independent of the syntactic modification.  This prediction was investigated in this Direct Modification experiment.
In this experiment, the position of the critical noun in the sentence was varied, and the noun was always directly modified by the adjective big or small. The critical nouns were always subordinate referent labels. To create respective sentences, a second noun was used which described a visually salient feature of the referent. For example, the referents for one of the dog contexts were prize-winners, as indicated by prize-bows depicted on the referents. So the critical sentence was either “That prize-winner is a big Great Dane” (predicate-N) or “That big Great Dane is a prize-winner” (subject-N). 
The referents appeared in a basic-level context, which included two other members of the same subordinate category as the referent, and two other individuals with the feature described by the second noun of the sentence, e.g. in the dog-context there were two other prize-winners. Because the reference-predication trade-off is based on explaining away a noun via its potential referential use, through this context manipulation the referential utilities of the two nouns of the sentence is equated, such that only the noun’s syntactic position and combination with the deictic ‘that’ could provide a cue towards referential use. Therefore the critical subordinate noun is expected to constrain the inferred comparison class more strongly when it appears in the predicate of the sentence than in the subject.   
The experimental set-up was similar  to the set-up of experiment 3. Five different contexts were used in this experiment: there were two dog contexts, a flower, a bird and a tree context (Table 2). Four out of five contexts were sampled for each participant.  Participants completed two experimental blocks, each consisting of warm-up and main trials using two of the sampled categories. In the first block, participants first completed three rounds of labeling warm-up trials. A round consisted of a demonstration trial where participants saw two subordinate members of a basic-level category used in this block and read their labels. For example, they saw pictures of a sunflower and a daisy next to each other and read “This is a sunflower” and “This is a daisy”, respectively. They could proceed after 3.5 seconds to the next trial where they had to label other instances of the same categories themselves. They also had to provide a common label for the pictures (i.e. flowers). The order of the pictures was randomized between-participants. They were provided feedback on their labels and could proceed only after correcting their labels.  After two labeling warm-up rounds, participants completed two demonstration trials of at least 3.5 seconds each learning about the additional features of the referents described by the second noun of the critical sentences in main trials. For example, participants saw a picture depicting the sunflower and the daisy in pots with bows, and read: “These flowers are gifts. Notice the bow on the pots.”. Finally, participants completed a comparison class paraphrase practice trial, identical to the one used in experiment 3. The warm-up trials in the second experimental block were identical, but there was no paraphrase practice trial.   
Then, participants completed four trials - two main and two filler trials, in randomized order, where a filler trials was always the first trial of the block. In the main trials, a subordinate referent with an additional feature (e.g. a prize-winner bow) appeared in the corresponding context, as described above. Participants read different context stories for each context (table 2).  For example, for a flower context, they read “You and your friend are at their garden and you see the following:” above the context picture. Below, they read “Your friend runs far ahead of you. You see your friend in the distance:”, followed by a depiction of the referent with the additional feature next to a person; to induce the illusion of distance, both were small relative to the context picture. Then they read “Your friend says:”, followed by the critical sentence. Finally, they were asked: “What do you think your friend is saying it is {big, small} relative to?”, introducing the paraphrase template, like in experiment 3. For a given category, one of the possible targets appeared in this critical trial (e.g. the sunflower). The other possible target (i.e. the dandelion) then appeared in a filler trial in the same block. Filler trials were identical to main trials with basic-level contexts from experiment 3. The size of referent (i.e. large-subordinate vs. small-subordinate) was counterbalanced across syntactic conditions and trial types within-participant, resulting in 8 unique conditions. Each participant saw each condition once, resulting in eight main trials.

\subsection{Participants:}
\subsection{Results:}


\chapter{A Bayesian Reference-Predication Model}
\section{Understanding Rational Speech Act Models}
\pt{tbd}
\section{Refpred-RSA}
\pt{tbd}

\chapter{Discussion}
Interface between syntax, semantics and pragmatics

definiteness confound in experiments

future work: more adjectives, prosody, other pragmatic frames, stronger referential pressure

developmental implications / anna's work and results

use different targets, gather typicality judgements, real-world pictures?

corpus analysis?

first model to our knowledge incorporating speakers pursuing several communicative goals. 

first experimental data showing the context dependence of adjectives: same utterances are actually interpreted differently in distinct contexts

first work to attempt to integrate influences from different sources of information for the comparison class

\pt{discuss particular operationalization of E4 here or in the experimental chapter} 

connection to other context-depending linguistic phenomena; case study for context-dependence and vagueness

\pt{discuss in chapter 6 that deconfounding definiteness from syntactic manipulation should be addresses in future research; keep it maximally symmetric in E1-3; tentative predictions for E4: same distinction for "A prize-winner is a big great dane" vs "A big great dane is a prize-winner"; infelicitous presuppositions for both parts being definite; also discuss connection to plural / generics / predagogical language;}

\chapter*{Declaration}
I declare that..

\appendix
\chapter{Appendix}	
\section{Experimental Materials}
\subsection{Bot-check Trial}
The names used in the bot-check trials were: 
\begin{itemize}
\item Male names: James, John, Robert, Michael, William, David, Richard, Joseph, Thomas, Charles 
\item Female names: Mary, Patricia, Jennifer, Linda, Elizabeth, Barbara, Susan, Jessica, Sarah, Margaret. 
\end{itemize}
This trial view was developed and provided by Elisa Kreiss. 
\subsection{E2 Exclusion Criteria}
In the Sentence Rating Experiment (E2), data from \rlgetvariable{myvars-rating.csv}{nExcludedTotal} participants was excluded. \rlgetvariable{myvars-rating.csv}{nNonEN} indicated a native language other than English. Data from \rlgetvariable{myvars-rating.csv}{nFailedWarmUp} was excluded due to failed warm-up trials. This means, participants provided a lower rating of the sentence “The chair is blue” than the sentence “The chair is yellow” on the chair warm-up trial; it was also counted as a fail if participants rated the sentence “The basketball is green” higher than the sentence “The basketball is orange”, or if the rating of the sentence “The basketball is orange” received a rating of less than 50 on the basketball trial.

Furthermore, data from \rlgetvariable{myvars-rating.csv}{nFailedMains} participants were excluded who provided the same ratings within 5 points for one syntactic condition on every trial (one of the sentences on every trial), or those who provided the same ratings of the two sentences on every trial. 
However, choosing exclusion criteria based on participants’ performance in the main trials might have been an overly restrictive or biasing criterion. So an exploratory analysis was conducted on the full dataset, where participants were only excluded based on their performance in the practice trials. 
This exploratory analysis revealed results qualitatively and quantitatively very similar to results from the main preregistered analysis: participants dispreferred sentences with a subordinate predicate noun, compared to sentences with basic-level subordinate nouns, but did not show any preferences in the subject-noun condition (syntax-by-noun interaction: $\beta = \rlgetnum{expt1_brm_full_exploratory.csv}{Rowname}{syntax:NP}{Estimate}{2}  [\rlgetnum{expt1_brm_full_exploratory.csv}{Rowname}{syntax:NP}{l.95..CI}{2}, \rlgetnum{expt1_brm_full_exploratory.csv}{Rowname}{syntax:NP}{u.95..CI}{2}]$). They also overall preferred the subject-N syntax ($\beta = \rlgetnum{expt1_brm_full_exploratory.csv}{Rowname}{syntax}{Estimate}{2} [\rlgetnum{expt1_brm_full_exploratory.csv}{Rowname}{syntax}{l.95..CI}{2}, \rlgetnum{expt1_brm_full_exploratory.csv}{Rowname}{syntax}{u.95..CI}{2}] $), as well as basic-level nouns ($\beta = \rlgetnum{expt1_brm_full_exploratory.csv}{Rowname}{NP}{Estimate}{2} [\rlgetnum{expt1_brm_full_exploratory.csv}{Rowname}{NP}{l.95..CI}{2},\rlgetnum{expt1_brm_full_exploratory.csv}{Rowname}{NP}{u.95..CI}{2}] $). 

\printbibliography
%\bibliography{references}
\end{document}
