The meaning of natural language expressions heavily depends on the context in which these expressions are used, but speakers rarely explicitly outline which aspects of the context are relevant for their interpretation. 

This issue is clearly illustrated by utterances involving gradable adjectives like \textit{big, small, tall, expensive} etc. These adjectives are typically taken to describe a \emph{degree} to which an object possesses some property, e.g., the degree of bigness (i.e., size) for the adjective \emph{big}, but specific degrees a speaker intends to convey vary a lot depending on the particular referent and context. Intuitively, the utterance “That’s big!” denotes quite different size degrees, depending on whether it was uttered in reference to a flower or in reference to a house, while both objects could potentially co-occur in the same perceptual context; given this utterance, it is left to the listener to identify the correct referent and size degree. The aspect that goes unsaid and that allows for the flexible use of the adjective \textit{big} across referents and contexts is \textit{what the intended referent is big relative to}. Humans easily infer that two objects might be compared to different things: for instance, it is more likely that the flower in the previous example is big for its kind of flowers or relative to other flowers around it, whereas the house is probably rather being compared to other houses in the neighborhood. 

However, speakers rarely explicitly state a referent's comparison class---the set of entities the target is compared against---and it is left to the addressee to establish the relevant comparison class \parencite{Solt2009}. Listeners have vast general knowledge and experience about the world helping them interpret context-sensitive language \parencite{tessler2017warm}, but it is unclear what additional \emph{linguistic} features listeners attend to when resolving context-sensitive ambiguity. 
In particular, previous work has left open the question how listeners establish a comparison class in order to interpret a gradable adjective, given infinitely many a priori conceivable options for the comparison class.

This work aims to fill that gap by investigating the role of syntactic structure for sentences containing gradable adjectives, and suggests that syntax provides a cue to contextually relevant aspects for adjective interpretation, which listeners integrate with other cues like perceptual context and world knowledge.\footnote{This thesis summarizes and extends the work by Tessler, Tsvilodub, Snedeker and Levy published in \textcite{tessler2020}, that appeared in the \textit{Proceedings of the 42nd Annual Meeting of the Cognitive Science Society}.} In particular, we hypothesize that syntactic structure reflects informational goals interlocutors strive to achieve; listeners reason about these goals pragmatically when inferring the comparison class of gradable adjectives. Focusing on the informational goals of \textit{reference} and \textit{predication}, this work presents a novel \textbf{reference-predication trade-off hypothesis} of comparison class inference, contributing to the body of research on gradable adjectives and providing a case study for the relationship between linguistic form and meaning. Evidence from four behavioral experiments is provided in support of this functional hypothesis, as well as a Bayesian model of gradable adjective interpretation, which shows that sophisticated pragmatic reasoning about syntactic structure can be captured using the generic probabilistic \emph{Rational Speech Act} framework \parencite{goodman2016}. 
	
