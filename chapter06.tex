\pt{Bullshit yet} Humans use language across infinitely many different situations. During infancy already they learn that the same words might be used in different contexts and convey meanings, producing a fascinatingly versatile and efficient language. 
Relative gradable adjectives are one, but not the only, example of natural language expressions whose meaning greatly depends on context. Yet speakers almost never establish explicitly which aspects of context are relevant and leave it to the listener to pragmatically reconstruct.  Understanding gradable adjectives requires establishing a context-dependent comparison class. Comparison classes are also relevant for understanding other kinds of context-sensitive language like quantifiers \parencite{scholler2017semantic} or generics \parencite{tessler2019language}. Other instances of context-sensitive language involve phenomena like ....
This work investigated aspects of interpretation of relative gradable adjectives as a case-study of context-sensitive language. In particular, this work outlined a novel functional hypothesis regarding the role a noun might habe on comparison class inference. The main proposed inference highlighted by four behavioural experiments is that participants are less likely to take the noun as a cue towards the comparison class when the noun can be expained as intended for reference. 
This hypothesis was tested in four experiments. Experiment 1 showed that participants appreciably disprefer sentences where a subordinate noun appeared in the predicate compared to sentences with a basic-level in the predicate, indicating that participants preferred sentences with a more felicitous comparison class label in the predicate. Experiment 2 revealed that participants flexibly adjust their noun choices given different syntactic frames in a free-production setting, producing more basic-level nouns in the predicate than in the subject position. Experiment 3 indicated that participants are highly sensitive to the perceptual context of the utterance, to the noun of the utterance and to its syntactic position, trading-off its utility in reference and predication when reasoning about the comparison class. Finally, a pilot study for Experiment 4 showed that reasoning about informational goals accomplished by the nouns is indeed the driver behind participants' reasoning about the cue strength of the noun towards the comparison class more so than presence of direct syntactic modification of the noun by the adjective. Participants were more likely to infer the subordinate comparison class from directly modified subordinate nouns appearing in the predicate than in the subject position. 
Together, these experiments provide converging evidence that humans use information-structural when reasoning about the comparison class, consistent with the reference-predication trade-off hypothesis. 

\pt{discuss particular operationalization of E4 here or in the experimental chapter} 
Novel contribution of this experimental data. first experimental data showing the context dependence of adjectives: same utterances are actually interpreted differently in distinct contexts

Experiments presented here investigated the role of perceptual context, noun type and syntactic position on comparison class inferences. However, all experiments presented critical utterances in written form, and relied on an operationalization of the two informational goals via the syntactic manipulation between subject and predicate position of the noun. One other factor that is closely related to information-structural phenomena like focus is \emph{prosody} \parencite{krifka2008basic}. In particular, uttering the sentence "That Great Dane is \emph{big}" (\emph{big} being focused) might convey that the speaker deems the size of this referent particularly noteworthy and therefore shifts the communicative goal towards predication - a perfectly reasonable scenario where the Great Dane is actually big \emph{for} a Great Dane, making the subordinate subject-noun the comparison class.  Alternatively, the same scenario is imaginable in a situation where someone said "That Great Dane is small" and someone what reply "No, that Great Dane is big". The goes in the direction of meta-linguistic / predagogical uses of gradable adjectives. For the utterance "That's a big Great Dane", also prosodically different readings seem plausible in different contexts. 
All these examples seem consistent with the reference-predication hypothesis.
The effects of prosody on comparison class inferences should be addressed in future work. 

Another potential confound - definiteness. \pt{discuss in chapter 6 that deconfounding definiteness from syntactic manipulation should be addresses in future research; keep it maximally symmetric in E1-3; tentative predictions for E4: same distinction for "A prize-winner is a big great dane" vs "A big great dane is a prize-winner"; infelicitous presuppositions for both parts being definite; also discuss connection to plural / generics / predagogical language;}

Use more adjectives, different pictures, consider typicality effects. Investigate referential pressure. 

Model stuff. 
Novel contribution of this model.   
first model to our knowledge incorporating speakers pursuing several communicative goals. 


Developmental implications. Anna's work.

Conclusion: Interface between syntax, semantics and pragmatics. first work to attempt to integrate influences from different sources of information for the comparison class. connection to other context-depending linguistic phenomena; case study for context-dependence and vagueness
