Humans use language across infinitely many different situations. By the age of two infants already learn that the same words might be used in different contexts and convey different meanings, allowing for fascinatingly versatile and efficient language use \parencite{Mintz2002, ebeling1994children}. 
Relative gradable adjectives are one, but not the only, example of natural language expressions whose meaning greatly depends on context (e.g., next to \textcite[indexicals][]{braun2017}; or \textcite[anaphoras][]{goldberg2017one}, inter alia). Yet speakers almost never establish explicitly which aspects of context are relevant and leave it to the listener to pragmatically reconstruct. Interpreting gradable adjectives requires establishing a context-dependent comparison class by inferring to what aspects of (non-)linguistic context the referent is being compared. This task is relevant for the interpretation of a number of other expressions, as comparison classes are also relevant for understanding other kinds of context-sensitive language like quantifiers \parencite[e.g., "John ate \emph{many} of hot dogs",][]{scholler2017semantic} or generics \parencite[e.g., "Dogs are friendly" \emph{[relative to other animals]},][]{tessler2019language}. 

This thesis investigated the interpretation of relative gradable adjectives as a case-study of context-sensitive language and attempted to contribute to understanding how exactly listeners might establish relevant comparison classes. 
Standard literature on gradable adjective semantics suggests that the comparison class might be supplied by the noun the adjective combines with, or by relevant contextual aspects; however, this work argued that purely compositional accounts might be too rigid to explain the flexible use of gradable adjectives across contexts. Furthermore, much of the research on gradable adjective semantics eschewed the question how exactly a comparison class might be determined contextually, instead focusing on its integration into compositional semantics \parencite{Kennedy2007, kennedy2012, Kamp1975, Cresswell1976, Solt2009}. 
Therefore, this work outlined a novel functional hypothesis regarding the role a noun might have for comparison class inference in addition to contextual cues --- namely the \emph{reference-predication trade-off hypothesis}. The quintessential prediction derived from this hypothesis is that listeners are less likely to take the noun as a cue towards the comparison class when the noun can be expained away as intended for the informational goal of reference, operationalized via the syntactic position of the noun the adjective combines with. Specifically, nouns appearing in the subject of the sentence can be explained by their utility in reference and are therefore less likely to set the comparison class, while nouns in the predicate are less likely to be referential and are therefore more likely to constrain the comparison class.

Four experiments were conducted to investigate this hypothesis. Experiment 1 showed that participants appreciably disprefer sentences where a subordinate noun appeared in the predicate compared to sentences with a basic-level noun in the predicate as describing a normal-sized referent, indicating that participants preferred sentences with a more felicitous comparison class label in the predicate. Experiment 2 revealed that participants flexibly adjust their noun choices given different syntactic frames in a free-production setting, producing more basic-level nouns in the predicate than in the subject position. Experiment 3 indicated that participants are highly sensitive to the perceptual context of the utterance, to the noun of the utterance and to its syntactic position when asked to paraphrase the comparison class, trading-off the noun's utility in reference and predication when reasoning about its cue strength towards the comparison class. Finally, a pilot study for Experiment 4 showed that reasoning about informational goals accomplished by the noun is indeed the driver behind participants' reasoning about comparison classes, as opposed to mere presence of direct syntactic modification of the noun by the adjective. % Participants were more likely to infer the subordinate comparison class from directly modified subordinate nouns appearing in the predicate than in the subject position. 
Together, these experiments provide converging evidence that humans use information structure when reasoning about the comparison class, consistent with the reference-predication trade-off hypothesis. 

Experiments presented here investigated the role of perceptual context, noun type and syntactic position on comparison class inferences; while covering major cues towards the comparison class, the particular operationalization of the hypothesis via the syntactic manipulation between subject and predicate position of the noun invites investigation of further aspects in future studies. For one, all experiments presented critical utterances in written form, yet one other factor that is closely related to information-structure is \emph{prosody} \parencite{krifka2008basic}. Prosody might provide a strategy to both structure information and construct new content: for instance, prosody realising focus of a sentence might indicate the presence of relevant alternatives for the focused expression, convey the topic of the utterance or emphasize new information \parencite{krifka2008basic, selkirk1995sentence}. 
Applied to sentences used in present studies, uttering the sentence "That Great Dane is \textsc{big}" (\textsc{big} being prosodically prominent) might convey that the speaker deems the size of this referent particularly noteworthy and therefore shifts the communicative goal towards predication - a perfectly reasonable scenario where the Great Dane is actually big \emph{for} a Great Dane, making the subordinate subject-noun the comparison class.  Alternatively, the same scenario is imaginable in a situation where someone said "That Great Dane is small" and the speaker replied "No, that Great Dane is \textsc{big}". The latter example goes in the direction of meta-linguistic or predagogical uses of gradable adjectives where the speaker informs the listener about what he considers a viable use of size adjectives given the particular referent. Uttering "That \textsc{Great Dane} is big" might signal that a Great Dane is being contrasted against other categories or referents, highlighting the referential goal of the noun.
For the predicate-N utterance prosodically different readings also seem possible in different contexts. For instance, uttering "\textsc{That}'s a big Great Dane" might contrast a particular referent against other Great Danes; or teach the listener about what the speaker considers a good representative of a big Great Dane. The latter example seems plausible in a context where e.g. other big dogs or Great Danes were previously discussed. Uttering "That's a \textsc{big} Great Dane" again seems to imply meta-linguistic or contrastive intentions, while "That's a big \textsc{Great Dane}" would highlight the noun as signalling the comparison class. 
Generally, prosody as a tool for both information packaging and content construction might affect the question under discussion and therefore the informational goal in focus, so effects of prosody on comparison class inferences and their connection to the refernence-predication hypothesis should be addressed in future work \parencite[cf.][]{krifka2008basic}. 

For another thing, it should be noted that this particular experimental approach was chosen in order to keep a maximally simple and controlled design. So for practical reasons of stimulus presentation only size-adjectives 'big' and 'small' were used; however, the reference-predication trade-off hypothesis is applicable to relative gradable adjectives in general, so further experiments involving other adjectives should be conducted. 
Furthermore, the experiments used stimuli from natural basic-level categories only; salient taxonomic representation of such categories might play a role for listeners' reasoning about potential comparison classes, by providing salient viable alternative comparison categories (i.e., subordinate and basic-level categories, \textcite[cf.][]{rosch1976, tenenbaum2011grow}); there might be potentially relevant psychological differences between natural and articifical concepts though%have been addressed in the literature
 \parencite{kalish2002gold}. 
Additionally, only one specific (schematic) referent picture per subordinate category was used across experiments; future research should consider varying the pictures in order to average out potential typicality and nameability effects of these pictures as representations of those categories. Relatively high by-target random intercepts across experiments suggested that targets might have varied in those aspects (e.g., resulting in different propensities of the speakers to use subordinate labels). Running a typicality or nameability rating study on the used categories and pictures might also be useful for eliciting priors for fitting the RSA model quantitatively to observed data \parencite{franke2016does}. %Additionally, the relation of comparison classes to typicality of the target for its basic-level category should be investigated more carefully. 

Furthermore, in order to be able to create maximally symmetric utterances across different contexts and syntactic conditions, the noun phrase in the predicate position was always indefinite, and the subject noun phrase was always definite. If the predicate was definite, the triggered uniqueness presupposition would have been violated in subordinate contexts, deeming these sentences generally less natural (i.e., saying "That's \emph{the} big Great Dane" would trigger the expectation that there is only one Great Dane, which is not true of contexts used in the experiments) \parencite[cf.][]{syrett2010meaning}. 
For this reason, the definiteness of the noun is perfectly confounded with its position throughout the experiments. Yet given an appropriate discourse context where there is only one individual denoted by the noun utterances with a definite predicate noun could be tested. The reference-predication hypothesis would predict that definite nouns in the predicate might still be more likely to establish the comparison class than subject nouns irrespective of their higher referential value compared to indefinite ones, as long as reference was already established in the subject. 
The hypothesis, however, does not seem to be operationalized by alternative sentences where the subject noun would be indefinite, since indefinite nouns generally do not refer; these sentences might rather interface with generic or meta-linguistic uses of adjectives \parencite{Reboul2001, tessler2019language, barker2002dynamics}. 

Generally, referential pressure was not very high in current experiments. Referential purposes of the noun were mainly communicated through its combination with the deictic, and its position in the subject which might rely on general information-structural expectation (s. Chapter \ref{chapter03}). The target referent was perceptually highlighted in a separate picture throughout the experiments. The contextual referential pressure might be potentially manipulated in further experiments, e.g., by presenting the referent as a non-highlighted member of the context. 
%Another potential confound - definiteness. \pt{discuss in chapter 6 that deconfounding definiteness from syntactic manipulation should be addresses in future research; keep it maximally symmetric in E1-3; tentative predictions for E4: same distinction for "A prize-winner is a big great dane" vs "A big great dane is a prize-winner"; infelicitous presuppositions for both parts being definite; also discuss connection to plural / generics / predagogical language;}

%Use more adjectives, different pictures, consider typicality effects.  namability effects 
%Investigate referential pressure.
Besides experimental evidence, this work also provides a computational model of the reference-predication hypothesis within the Rational Speech Act framework. The proposed model builds upon and advances beyond previous RSA-models of gradable adjectives by incorporating reasoning about perceptual, lexical and syntactic cues towards the comparison class. That is, the pragmatic listener model infers a potential state of the world (a referent and its size) along with the likely comparison class. In particular, the listener reasons about how a speaker would behave in order to communicate a specific state, also considering their prior knowledge about likely sizes of different categories. Crucially, the listener assumes that the speaker might variably choose the intended adjective meaning, and therefore the comparison class. 
In addition, the speaker constructs her utterances incrementally by deciding whether to put the noun in the utterance subject or in the predicate, so as to optimally achieve reference with the subject, and predication with the predicate. This representation allows the speaker to achieve two informational goals simulatneously by using one sentence. By using hypothetic fixed priors, qualitative predictions were derived from the model. It was shown that it is a good first formalization of the reference-predication trade-off hypothesis because it captures essential contrasts observed in Experiment 3. Specifically, the pragmatic listener was more likely to take the noun as a cue towards the comparison class when it appeared in the predicate than in the subject, while taking into account referential utility of the noun. In sum, the proposed model provides a first attempt to integrate pragmatic reasoning about both syntax and semantics of an utterance within the Rational Speech Act framework. 
%Representing adjectival meaning as a local enrichment is novel in this adjective model. 

%\pt{Model discussion and criticism, Novel contribution of this model if any.   }
While the proposed RSA model provides a good first approximation of the reference-predication hypothesis, the derived predictions were only qualitative. Future work should evaluate its potential to fit the data observed in Experiment 3 quantitatively. To do so, experiments eliciting participants' prior knowledge of relative category sizes could be conducted \parencite{franke2016does}, or the priors could be reconstructed from indirect experimental results relying on the same prior knowledge \parencite[following][]{tessler2017warm}. The free speaker optimality paramter should then also be determined more accurately, e.g., via a maximum a posteriori estimate. %Furthermore, the model represented a higher referential pressure on the speaker than might have been the case in the experimental set-up. 
Furthermore, additional assumptions were made in the model: for one, the inference over the standard of comparison was performed by the literal listener. The cognitive plausibility of this adjective meaning representation remains to be investigated. In addition, it was assumed that the listener and the speaker share identical world knowledge about likely sizes for different categories. Incorporating their reasoning about each other's world knowledge in future work would potentially extend this model to account for meta-linguistic uses of gradable adjectives (s. Chapter \ref{chapter02}; \textcite{barker2002dynamics}). Lastly, the model utilized uniform priors over potential referents in context, as well as over potential comparison classes. Future work should revise these assumptions because referents might differ in their perceptual saliency --- e.g., referents which stand out with their size might be generally more likely to be talked about; and comparison class avaliability might vary across different subordinate categories due to namability or typicality effects. One possibility would be to consider corpus frequencies of respective nouns as an approximation of the prior \parencite[following the model by][]{tessler2017warm}. Generally, the model provides the first qualitative example of context-sensitive language interpretation unifying reasoning about several cues to the intended meaning. The hope is to supply a basis for future investigations of such holistic RSA models applied to various cases of vague expressions, in addition to providing a more sophisticated model of gradable adjective understanding.
%first model to our knowledge incorporating speakers pursuing several communicative goals. 
%Qualitative fit, quantitative fit?

Cues that contribute to adult gradable adjective interpretation also have implications for investigating the developmental course of children's understanding of complex context-sensitive expressions. In particular, already by the age of two infants understand adjectives like 'big' and 'small' and appreciate their context-sensitive nature, sometimes even without adults specifically pointing out the relevant features of context \parencite{Mintz2002, ebeling1994children}. That is, the critical skill to acquire is establishing the correct comparison class; while adults often use prepositional \emph{for}-phrases in child-directed speech, the phrases are often indicative of a particular kind of gradable adjective uses - the functional uses (s. Chapter 2, \textcite{ebeling1994children}). Yet when not producing a \emph{for}-phrase, adults might include other syntactic cues towards the comparison class compensating for more ambiguous adjective uses. 

Taking a developmental perspective on the reference-predication hypothesis, one might derive the prediction that adults use nouns in the predicate of sentences more often than in the subject in order to restrict the comparison class more strongly and make the adjective interpretation easier for children. A study was conducted wherein the Providence corpus was annotated with respect to linguistic and environmental cues available in the recorded contexts \parencite{sinelnikova2020}. It provides preliminary evidence that the majority of recorded uses of the adjective 'big' were indeed prenominal cases where the modified nominal appears in the predicate of the sentence \parencite{sinelnikova2020}. 
Moreover, it showed that more modified nouns appeared in the predicate when there were less distractors in the context; when there were more perceptual distractors the modified noun was rather placed in the subject, contributing to reference. Furthermore, it was found that the referent was mostly physically copresent when the noun was used in the predicate, suggesting that reference might have been established by means other than the noun (e.g., pointing). Finally, the intended comparison class for the adjective-nominal predicate constructions was overwhelmingly the normative comparison class (i.e., making reference to implicit general knowledge of the abstract basic-level  category of the referent, mostly denoted by the noun; s. Chapter \ref{chapter02}), indicating that these syntactic frames were used when establishing the comparison class might have been more challenging for the children. These preliminary results are consistent with predictions of the reference-predication trade-off hypothesis ---  nouns in the predicate seem to communicate the comparison class of a referent, facilitating understanding of the gradable adjective for kids who might lack other means for pragmatic reasoning like substantial world knowledge. 
Understanding what kinds of cues are available to children in naturalistic settings might provide a starting point for investigating how they succeed in learning context-sensitive expressions. 

%highlighting the complex relation of an utterance meaning and its form.  Expt / modelling results, new hypothesis% --- investigating human reasoning about this connection provides valuable insights for computational modeling of vague language.

%Sum up uitlity of the reference-predication hypothesis
%copula
%cross-linguistic ideas / reviewer's comments

\section{Conclusion}
% Interface between syntax, semantics and pragmatics. first work to attempt to integrate influences from different sources of information for the comparison class. connection to other context-depending linguistic phenomena; case study for context-dependence and vagueness
To sum up, this thesis took a step towards understanding how interlocutors flexibly use relative gradable adjectives across contexts, by presenting a novel reference-predication trade-off hypothesis of comparison class inference. 
This hypothesis provides a holistic account of how %interlocutors might establish comparison classes when using relative gradable adjectives. It was shown that 
humans might integrate various cues and reason about referential utility of the noun in context, trading it off with the noun's utility in communicating the comparison class. This inferential account is supported by converging evidence from various experiments. Finally, the hypothesis was formalized computationally within the Rational Speech Act framework, utilizing domain-general Bayesian inference tools widely used in models of cognition. % shown that domain-general Bayesian reasoning is a powerful tool suitable to capture this complex inferential account computationally.  %able to account for vague language interpretation. 
 %Integrating evidence from different information sources, 
This work highlighted the sophisticated reasoning at the interface of syntax, semantics and pragmatics humans perform in order to make clear use vague language. %establish comparison classes. %This case-study of vague language provides a fascinating example for how humans can accomplish big goals by harnessing a small but powerful set of pragmatic reasoning skills. 
