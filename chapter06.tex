Interface between syntax, semantics and pragmatics

definiteness confound in experiments

future work: more adjectives, prosody, other pragmatic frames, stronger referential pressure

developmental implications / anna's work and results

use different targets, gather typicality judgements, real-world pictures?

corpus analysis?

first model to our knowledge incorporating speakers pursuing several communicative goals. 

first experimental data showing the context dependence of adjectives: same utterances are actually interpreted differently in distinct contexts

first work to attempt to integrate influences from different sources of information for the comparison class

\pt{discuss particular operationalization of E4 here or in the experimental chapter} 

connection to other context-depending linguistic phenomena; case study for context-dependence and vagueness

\pt{discuss in chapter 6 that deconfounding definiteness from syntactic manipulation should be addresses in future research; keep it maximally symmetric in E1-3; tentative predictions for E4: same distinction for "A prize-winner is a big great dane" vs "A big great dane is a prize-winner"; infelicitous presuppositions for both parts being definite; also discuss connection to plural / generics / predagogical language;}